
%% bare_jrnl.tex
%% V1.4b
%% 2015/08/26
%% by Michael Shell
%% see http://www.michaelshell.org/
%% for current contact information.
%%
%% This is a skeleton file demonstrating the use of IEEEtran.cls
%% (requires IEEEtran.cls version 1.8b or later) with an IEEE
%% journal paper.
%%
%% Support sites:
%% http://www.michaelshell.org/tex/ieeetran/
%% http://www.ctan.org/pkg/ieeetran
%% and
%% http://www.ieee.org/

%%*************************************************************************
%% Legal Notice:
%% This code is offered as-is without any warranty either expressed or
%% implied; without even the implied warranty of MERCHANTABILITY or
%% FITNESS FOR A PARTICULAR PURPOSE! 
%% User assumes all risk.
%% In no event shall the IEEE or any contributor to this code be liable for
%% any damages or losses, including, but not limited to, incidental,
%% consequential, or any other damages, resulting from the use or misuse
%% of any information contained here.
%%
%% All comments are the opinions of their respective authors and are not
%% necessarily endorsed by the IEEE.
%%
%% This work is distributed under the LaTeX Project Public License (LPPL)
%% ( http://www.latex-project.org/ ) version 1.3, and may be freely used,
%% distributed and modified. A copy of the LPPL, version 1.3, is included
%% in the base LaTeX documentation of all distributions of LaTeX released
%% 2003/12/01 or later.
%% Retain all contribution notices and credits.
%% ** Modified files should be clearly indicated as such, including  **
%% ** renaming them and changing author support contact information. **
%%*************************************************************************


% *** Authors should verify (and, if needed, correct) their LaTeX system  ***
% *** with the testflow diagnostic prior to trusting their LaTeX platform ***
% *** with production work. The IEEE's font choices and paper sizes can   ***
% *** trigger bugs that do not appear when using other class files.       ***                          ***
% The testflow support page is at:
% http://www.michaelshell.org/tex/testflow/



\documentclass[journal]{IEEEtran}
%
% If IEEEtran.cls has not been installed into the LaTeX system files,
% manually specify the path to it like:
% \documentclass[journal]{../sty/IEEEtran}




\usepackage{graphicx}
\usepackage{amsfonts}
\usepackage{amsmath} % assumes amsmath package installed
\DeclareMathOperator*{\argmax}{argmax}
\DeclareMathOperator*{\argmin}{argmin}
\usepackage{bm}

\usepackage{algorithmic}
\usepackage{algorithm}
\renewcommand{\algorithmicrequire}{\textbf{Input:}}
 \renewcommand{\algorithmicensure}{\textbf{Output:}}

\usepackage{lipsum}

\usepackage{threeparttable}
% Some very useful LaTeX packages include:
% (uncomment the ones you want to load)


% *** MISC UTILITY PACKAGES ***
%
%\usepackage{ifpdf}
% Heiko Oberdiek's ifpdf.sty is very useful if you need conditional
% compilation based on whether the output is pdf or dvi.
% usage:
% \ifpdf
%   % pdf code
% \else
%   % dvi code
% \fi
% The latest version of ifpdf.sty can be obtained from:
% http://www.ctan.org/pkg/ifpdf
% Also, note that IEEEtran.cls V1.7 and later provides a builtin
% \ifCLASSINFOpdf conditional that works the same way.
% When switching from latex to pdflatex and vice-versa, the compiler may
% have to be run twice to clear warning/error messages.






% *** CITATION PACKAGES ***
%
%\usepackage{cite}
% cite.sty was written by Donald Arseneau
% V1.6 and later of IEEEtran pre-defines the format of the cite.sty package
% \cite{} output to follow that of the IEEE. Loading the cite package will
% result in citation numbers being automatically sorted and properly
% "compressed/ranged". e.g., [1], [9], [2], [7], [5], [6] without using
% cite.sty will become [1], [2], [5]--[7], [9] using cite.sty. cite.sty's
% \cite will automatically add leading space, if needed. Use cite.sty's
% noadjust option (cite.sty V3.8 and later) if you want to turn this off
% such as if a citation ever needs to be enclosed in parenthesis.
% cite.sty is already installed on most LaTeX systems. Be sure and use
% version 5.0 (2009-03-20) and later if using hyperref.sty.
% The latest version can be obtained at:
% http://www.ctan.org/pkg/cite
% The documentation is contained in the cite.sty file itself.






% *** GRAPHICS RELATED PACKAGES ***
%
\ifCLASSINFOpdf
  % \usepackage[pdftex]{graphicx}
  % declare the path(s) where your graphic files are
  % \graphicspath{{../pdf/}{../jpeg/}}
  % and their extensions so you won't have to specify these with
  % every instance of \includegraphics
  % \DeclareGraphicsExtensions{.pdf,.jpeg,.png}
\else
  % or other class option (dvipsone, dvipdf, if not using dvips). graphicx
  % will default to the driver specified in the system graphics.cfg if no
  % driver is specified.
  % \usepackage[dvips]{graphicx}
  % declare the path(s) where your graphic files are
  % \graphicspath{{../eps/}}
  % and their extensions so you won't have to specify these with
  % every instance of \includegraphics
  % \DeclareGraphicsExtensions{.eps}
\fi
% graphicx was written by David Carlisle and Sebastian Rahtz. It is
% required if you want graphics, photos, etc. graphicx.sty is already
% installed on most LaTeX systems. The latest version and documentation
% can be obtained at: 
% http://www.ctan.org/pkg/graphicx
% Another good source of documentation is "Using Imported Graphics in
% LaTeX2e" by Keith Reckdahl which can be found at:
% http://www.ctan.org/pkg/epslatex
%
% latex, and pdflatex in dvi mode, support graphics in encapsulated
% postscript (.eps) format. pdflatex in pdf mode supports graphics
% in .pdf, .jpeg, .png and .mps (metapost) formats. Users should ensure
% that all non-photo figures use a vector format (.eps, .pdf, .mps) and
% not a bitmapped formats (.jpeg, .png). The IEEE frowns on bitmapped formats
% which can result in "jaggedy"/blurry rendering of lines and letters as
% well as large increases in file sizes.
%
% You can find documentation about the pdfTeX application at:
% http://www.tug.org/applications/pdftex





% *** MATH PACKAGES ***
%
%\usepackage{amsmath}
% A popular package from the American Mathematical Society that provides
% many useful and powerful commands for dealing with mathematics.
%
% Note that the amsmath package sets \interdisplaylinepenalty to 10000
% thus preventing page breaks from occurring within multiline equations. Use:
%\interdisplaylinepenalty=2500
% after loading amsmath to restore such page breaks as IEEEtran.cls normally
% does. amsmath.sty is already installed on most LaTeX systems. The latest
% version and documentation can be obtained at:
% http://www.ctan.org/pkg/amsmath





% *** SPECIALIZED LIST PACKAGES ***
%
%\usepackage{algorithmic}
% algorithmic.sty was written by Peter Williams and Rogerio Brito.
% This package provides an algorithmic environment fo describing algorithms.
% You can use the algorithmic environment in-text or within a figure
% environment to provide for a floating algorithm. Do NOT use the algorithm
% floating environment provided by algorithm.sty (by the same authors) or
% algorithm2e.sty (by Christophe Fiorio) as the IEEE does not use dedicated
% algorithm float types and packages that provide these will not provide
% correct IEEE style captions. The latest version and documentation of
% algorithmic.sty can be obtained at:
% http://www.ctan.org/pkg/algorithms
% Also of interest may be the (relatively newer and more customizable)
% algorithmicx.sty package by Szasz Janos:
% http://www.ctan.org/pkg/algorithmicx




% *** ALIGNMENT PACKAGES ***
%
%\usepackage{array}
% Frank Mittelbach's and David Carlisle's array.sty patches and improves
% the standard LaTeX2e array and tabular environments to provide better
% appearance and additional user controls. As the default LaTeX2e table
% generation code is lacking to the point of almost being broken with
% respect to the quality of the end results, all users are strongly
% advised to use an enhanced (at the very least that provided by array.sty)
% set of table tools. array.sty is already installed on most systems. The
% latest version and documentation can be obtained at:
% http://www.ctan.org/pkg/array


% IEEEtran contains the IEEEeqnarray family of commands that can be used to
% generate multiline equations as well as matrices, tables, etc., of high
% quality.




% *** SUBFIGURE PACKAGES ***
%\ifCLASSOPTIONcompsoc
%  \usepackage[caption=false,font=normalsize,labelfont=sf,textfont=sf]{subfig}
%\else
%  \usepackage[caption=false,font=footnotesize]{subfig}
%\fi
% subfig.sty, written by Steven Douglas Cochran, is the modern replacement
% for subfigure.sty, the latter of which is no longer maintained and is
% incompatible with some LaTeX packages including fixltx2e. However,
% subfig.sty requires and automatically loads Axel Sommerfeldt's caption.sty
% which will override IEEEtran.cls' handling of captions and this will result
% in non-IEEE style figure/table captions. To prevent this problem, be sure
% and invoke subfig.sty's "caption=false" package option (available since
% subfig.sty version 1.3, 2005/06/28) as this is will preserve IEEEtran.cls
% handling of captions.
% Note that the Computer Society format requires a larger sans serif font
% than the serif footnote size font used in traditional IEEE formatting
% and thus the need to invoke different subfig.sty package options depending
% on whether compsoc mode has been enabled.
%
% The latest version and documentation of subfig.sty can be obtained at:
% http://www.ctan.org/pkg/subfig




% *** FLOAT PACKAGES ***
%
%\usepackage{fixltx2e}
% fixltx2e, the successor to the earlier fix2col.sty, was written by
% Frank Mittelbach and David Carlisle. This package corrects a few problems
% in the LaTeX2e kernel, the most notable of which is that in current
% LaTeX2e releases, the ordering of single and double column floats is not
% guaranteed to be preserved. Thus, an unpatched LaTeX2e can allow a
% single column figure to be placed prior to an earlier double column
% figure.
% Be aware that LaTeX2e kernels dated 2015 and later have fixltx2e.sty's
% corrections already built into the system in which case a warning will
% be issued if an attempt is made to load fixltx2e.sty as it is no longer
% needed.
% The latest version and documentation can be found at:
% http://www.ctan.org/pkg/fixltx2e


%\usepackage{stfloats}
% stfloats.sty was written by Sigitas Tolusis. This package gives LaTeX2e
% the ability to do double column floats at the bottom of the page as well
% as the top. (e.g., "\begin{figure*}[!b]" is not normally possible in
% LaTeX2e). It also provides a command:
%\fnbelowfloat
% to enable the placement of footnotes below bottom floats (the standard
% LaTeX2e kernel puts them above bottom floats). This is an invasive package
% which rewrites many portions of the LaTeX2e float routines. It may not work
% with other packages that modify the LaTeX2e float routines. The latest
% version and documentation can be obtained at:
% http://www.ctan.org/pkg/stfloats
% Do not use the stfloats baselinefloat ability as the IEEE does not allow
% \baselineskip to stretch. Authors submitting work to the IEEE should note
% that the IEEE rarely uses double column equations and that authors should try
% to avoid such use. Do not be tempted to use the cuted.sty or midfloat.sty
% packages (also by Sigitas Tolusis) as the IEEE does not format its papers in
% such ways.
% Do not attempt to use stfloats with fixltx2e as they are incompatible.
% Instead, use Morten Hogholm'a dblfloatfix which combines the features
% of both fixltx2e and stfloats:
%
% \usepackage{dblfloatfix}
% The latest version can be found at:
% http://www.ctan.org/pkg/dblfloatfix




%\ifCLASSOPTIONcaptionsoff
%  \usepackage[nomarkers]{endfloat}
% \let\MYoriglatexcaption\caption
% \renewcommand{\caption}[2][\relax]{\MYoriglatexcaption[#2]{#2}}
%\fi
% endfloat.sty was written by James Darrell McCauley, Jeff Goldberg and 
% Axel Sommerfeldt. This package may be useful when used in conjunction with 
% IEEEtran.cls'  captionsoff option. Some IEEE journals/societies require that
% submissions have lists of figures/tables at the end of the paper and that
% figures/tables without any captions are placed on a page by themselves at
% the end of the document. If needed, the draftcls IEEEtran class option or
% \CLASSINPUTbaselinestretch interface can be used to increase the line
% spacing as well. Be sure and use the nomarkers option of endfloat to
% prevent endfloat from "marking" where the figures would have been placed
% in the text. The two hack lines of code above are a slight modification of
% that suggested by in the endfloat docs (section 8.4.1) to ensure that
% the full captions always appear in the list of figures/tables - even if
% the user used the short optional argument of \caption[]{}.
% IEEE papers do not typically make use of \caption[]'s optional argument,
% so this should not be an issue. A similar trick can be used to disable
% captions of packages such as subfig.sty that lack options to turn off
% the subcaptions:
% For subfig.sty:
% \let\MYorigsubfloat\subfloat
% \renewcommand{\subfloat}[2][\relax]{\MYorigsubfloat[]{#2}}
% However, the above trick will not work if both optional arguments of
% the \subfloat command are used. Furthermore, there needs to be a
% description of each subfigure *somewhere* and endfloat does not add
% subfigure captions to its list of figures. Thus, the best approach is to
% avoid the use of subfigure captions (many IEEE journals avoid them anyway)
% and instead reference/explain all the subfigures within the main caption.
% The latest version of endfloat.sty and its documentation can obtained at:
% http://www.ctan.org/pkg/endfloat
%
% The IEEEtran \ifCLASSOPTIONcaptionsoff conditional can also be used
% later in the document, say, to conditionally put the References on a 
% page by themselves.




% *** PDF, URL AND HYPERLINK PACKAGES ***
%
%\usepackage{url}
% url.sty was written by Donald Arseneau. It provides better support for
% handling and breaking URLs. url.sty is already installed on most LaTeX
% systems. The latest version and documentation can be obtained at:
% http://www.ctan.org/pkg/url
% Basically, \url{my_url_here}.




% *** Do not adjust lengths that control margins, column widths, etc. ***
% *** Do not use packages that alter fonts (such as pslatex).         ***
% There should be no need to do such things with IEEEtran.cls V1.6 and later.
% (Unless specifically asked to do so by the journal or conference you plan
% to submit to, of course. )


% correct bad hyphenation here
%\hyphenation{op-tical net-works semi-conduc-tor}


\begin{document}
%
% paper title
% Titles are generally capitalized except for words such as a, an, and, as,
% at, but, by, for, in, nor, of, on, or, the, to and up, which are usually
% not capitalized unless they are the first or last word of the title.
% Linebreaks \\ can be used within to get better formatting as desired.
% Do not put math or special symbols in the title.
\title{Exploring Manifold Learning For Automated Electrocardiography Arrhythmia Analysis}

%
%
% author names and IEEE memberships
% note positions of commas and nonbreaking spaces ( ~ ) LaTeX will not break
% a structure at a ~ so this keeps an author's name from being broken across
% two lines.
% use \thanks{} to gain access to the first footnote area
% a separate \thanks must be used for each paragraph as LaTeX2e's \thanks
% was not built to handle multiple paragraphs
%

\author{Yan~Yan,~\IEEEmembership{Student Member,~IEEE,}
       % x~x,~\IEEEmembership{Senior Member,~IEEE,}
        and~Lei~Wang,~\IEEEmembership{}% <-this % stops a space
\thanks{Y. Yan is with the Shenzhen Institutes of Advanced Technology, Chinese Academy of Sciences, and University of Chinese Academy of Sciences, China, (e-mail: yan.yan@siat.ac.cn).}% <-this % stops a space
\thanks{L. Wang is with Shenzhen Institutes of Advanced Technology, Chinese Academy of Sciences.}% <-this % stops a space
\thanks{Manuscript received ; revised. }}

% note the % following the last \IEEEmembership and also \thanks - 
% these prevent an unwanted space from occurring between the last author name
% and the end of the author line. i.e., if you had this:
% 
% \author{....lastname \thanks{...} \thanks{...} }
%                     ^------------^------------^----Do not want these spaces!
%
% a space would be appended to the last name and could cause every name on that
% line to be shifted left slightly. This is one of those "LaTeX things". For
% instance, "\textbf{A} \textbf{B}" will typeset as "A B" not "AB". To get
% "AB" then you have to do: "\textbf{A}\textbf{B}"
% \thanks is no different in this regard, so shield the last } of each \thanks
% that ends a line with a % and do not let a space in before the next \thanks.
% Spaces after \IEEEmembership other than the last one are OK (and needed) as
% you are supposed to have spaces between the names. For what it is worth,
% this is a minor point as most people would not even notice if the said evil
% space somehow managed to creep in.



% The paper headers
\markboth{}%
{xx \MakeLowercase{\textit{et al.}}: Exploring Manifold Learning Methods For Automated Arrhythmia Analysis}
% The only time the second header will appear is for the odd numbered pages
% after the title page when using the two side option.
% 
% *** Note that you probably will NOT want to include the author's ***
% *** name in the headers of peer review papers.                   ***
% You can use \ifCLASSOPTIONpeerreview for conditional compilation here if
% you desire.




% If you want to put a publisher's ID mark on the page you can do it like
% this:
%\IEEEpubid{0000--0000/00\$00.00~\copyright~2015 IEEE}
% Remember, if you use this you must call \IEEEpubidadjcol in the second
% column for its text to clear the IEEEpubid mark.



% use for special paper notices
%\IEEEspecialpapernotice{(Invited Paper)}




% make the title area
\maketitle

% As a general rule, do not put math, special symbols or citations
% in the abstract or keywords.
\begin{abstract}
Arrhythmia analysis had been a traditional research topic for decades. The various and changing morphologic features in the time series brought challenges for automated analysis. A geometrical descriptor for the time series is the mapping from the original input into some manifold. In this work, we investigate several manifold learning methods for electrocardiography embeddings, using the learned manifold features for arrhythmia analysis.  
The manifold methods have been validated on the MIT-BIH Arrhythmia Database for arrhythmia analysis. The intrinsic dimension analysis and the imbalance problem of electrocardiography dataset have been experimented and discussed.
The proposed experiments show that the manifold learning methods can generate meaningful representations for the arrhythmia classification.
However, the multi-layer autoencoder performs better than the other geometrical rule-based manifold learning methods.
The best results for this multi-class classification problem indicated an specificity of  98.30\% for ventricular ectopic beat class, and  84.71\% for supraventricular ectopic beat class, which is comparable to other similar tasks.
\end{abstract}

% Note that keywords are not normally used for peerreview papers.
\begin{IEEEkeywords}
Manifold Learning, Arrhythmia, ECG Classification, Machine Learning
\end{IEEEkeywords}






% For peer review papers, you can put extra information on the cover
% page as needed:
% \ifCLASSOPTIONpeerreview
% \begin{center} \bfseries EDICS Category: 3-BBND \end{center}
% \fi
%
% For peerreview papers, this IEEEtran command inserts a page break and
% creates the second title. It will be ignored for other modes.
\IEEEpeerreviewmaketitle



\section{Introduction}
% The very first letter is a 2 line initial drop letter followed
% by the rest of the first word in caps.
% 
% form to use if the first word consists of a single letter:
% \IEEEPARstart{A}{demo} file is ....
% 
% form to use if you need the single drop letter followed by
% normal text (unknown if ever used by the IEEE):
% \IEEEPARstart{A}{}demo file is ....
% 
% Some journals put the first two words in caps:
% \IEEEPARstart{T}{his demo} file is ....
% 
% Here we have the typical use of a "T" for an initial drop letter
% and "HIS" in caps to complete the first word.
\IEEEPARstart{T}{he} computer aided system for pathology detection and diagnosis are based on a feature space constructed from abundant clinical attributes.
The noninvasive, inexpensive and well-established technology of electrocardiographic signal in mobile health or personal health has the most significant popularity in heart function analysis. 
Automated arrhythmia analysis provides indispensable assist in long-term clinical monitoring, and a large number of approaches have been proposed for the task, easing the diagnosis of arrhythmic changes as well as further inspection, e.g., heart rate variability or heart turbulence analysis.
Therefore, information on electrocardiography signals related to the cardiac electrical activity would be represented by a large dimensional space from the data analysis angle.
Plenty of parameters have been extracted from the high dimensional data as the indicators for diagnosis. 
The high dimensional space hinders a proper interpretation of the embedded symbolic physiology in the feature space \cite{delgado2009dimensionality}.
However, the increasingly developing of the artificial intelligence domain and machine learning methods provides powerful tools to deal with the high dimensional electrocardiography data.



Lots of machine learning algorithms had been proposed for the classification and detection of electrocardiography signals. 
Linear discriminants \cite{chaza} and kNN \cite{melgan}, more complex classifiers like neural networks \cite{jiang, olmez, lin, osowski}, fuzzy inference engines \cite{osowski, kundu}, hidden Markov model \cite{andreao, coast}, independent component analysis \cite{zhu} and support vector machine  \cite{melgan, kampoura, khandoker} were also adopted by lots of researchers.  
ECG arrhythmia recognition systems highly depend on the determination of extracted electrocardiography features. 
Morphological features, frequency domain features, and statistical measures features for the six fundamental wave components (PQRSTU) are the most significant extracted features\cite{chia}. 
Morphological features include shapes, amplitudes, and durations were adapted primarily in \cite{jekova, christove, can}, frequency domain features like wavelet transformation were widely used \cite{inan}, \cite{banerjee} stationary features like higher-order statistics also had been developed. 
Principal component analysis \cite{stam} and Hermite functions \cite{lager} have been used in electrocardiography classification and related analysis technologies as well.
Almost every single published paper proposes a new set of features to be used, or a new combination of the existing ones \cite{mar}.


The results from these algorithms or models were not amenable to expert labelling, as well as for the identification of complex relationships between subjects and clinical conditions \cite{clifford}.
But for the ambulatory electrocardiography clinical application, as well as the normal application in daily healthcare monitoring for cardiac function or early warning of heart disease, an automated algorithm or model would have significant meaning.
The application of artificial intelligence methods has become an important trend in electrocardiography for the recognition and classification of different arrhythmia types \cite{clifford}. 
The data explosion puts forward the new request to the method of data processing and information mining.


Unsupervised learning-based approaches and the application to electrocardiogram classification in literatures mainly include clustering-based techniques \cite{lager, nishizawa, maier}, self-adaptive neural network-based methods \cite{palreddy, risk} and some hybrid unsupervised learning systems \cite{tadejko}. 
Recently, the developing of deep learning had proposed fruitful results in extracting representations from the original data space with large unlabelled data samples \cite{gogna2017semi}.

In deep learning, the neural networks based multiple levels of representations, which correspond to different levels of abstraction or concepts.
However, manifold learning, or dimensionality reduction, the methods come from the motivation that there might be meaningful structure of data which has fewer independent degrees of freedom than the input dimensionality \cite{tenenbaum2000global, roweis2000nonlinear}.
Here we use the low-dimensional descriptor or manifold features, while are computed with some data embedding or manifold learning methods.

The manifold learning algorithms are mainly differ in the way for computing manifold embeddings, usually they do preserve certain properties or characteristics of the original high-dimensional data space.
For example, the principal component analysis (PCA) attempts to preserve the data variance, multidimensional scaling attempts to preserve the Euclidean distance between all the original data points.
Generate embeddings based on the geometrical properties of data can provide meaning features for the classification tasks. 

Manifold learning had been successful used in facial expression recognition\cite{cheon2009natural}, image retrieval \cite{lin2005semantic}, ECG base ischemia analysis \cite{delgado2009dimensionality}, etc.
In this paper, we investigate several classical manifold learning methods for the application for arrhythmia analysis, which is based on ECG time series data samples.
We proposed the framework to embed ECG time series into some manifolds generating the embedded features. With a softmax classifier trained based on the embedded features, the classification task can be performed in the related manifold. Then the classification results can be used for arrhythmia analysis.

Besides, we consider autoencoder from the view point of manifold, but we tried to get the autoencoder embeddings in a quite different way from the others. An large unlabelled data set had been used in the training phase. This can be consider as some unknown geometrical rules which had been used as the optimization condition for autoencoder. Also, we use a layer-wised method for the training of multi-layer neural networks as the deep learning literature showed.

This organization of this paper is as follows. First we illustrated some manifold learning methods from Section II-A to Section II-G, then the softmax classifier was introduced. Then we described the experiments settings and some tricks should focus in manifold learning for ECG analysis in Section-III. Section IV shows the results and comparative analyses. while Section V concludes the paper.



%Manifold learning is a kind of method attempting to uncover the manifold structure in a data set, it can also be considered as a kind of nonlinear dimensional reduction technique.
%It has been widely used in multivariate data classification, data analysis, and data visualization.
%Manifold learning directly map the input data space into some feature space to find a better descriptor to analysis the real world problems. 

%Manifold learning directly map the input data space into some feature space to find a better descriptor to analysis the real world problems. 

%%%%%%%%%%%%%%%%%%%%%%%%%%%%%%%%%%%%%%%%%%%%%%%%%%%%%%%%%%%%%%
%
%                     SECTION III: Methodology
%
%%%%%%%%%%%%%%%%%%%%%%%%%%%%%%%%%%%%%%%%%%%%%%%%%%%%%%%%%%%%%%

\section{Methodology}
The arrhythmia analysis problem can be generally considered as one time series based classification problem from the viewpoint of data analysis.
Consider a set of $n$ electrocardiography based time series data set $\boldsymbol{X} = \{\boldsymbol{x}_i\} \subset \mathbb{R}^d$, is given as the samples.
Heartbeats in $\boldsymbol{X}$ include both labelled and unlabelled samples, the task is to estimate the labels of unlabelled data with some methods based on the labelled and relative methods.
Here we adopt several manifold learning based techniques to embed the high dimensional input data into some lower dimension subspace. 
With the techniques some data properties were preserved in the embedded subspace (feature space).
Then we train a classifier in the feature space using the labelled data, and use the classifier to classify the unlabelled data. From which we can get the arrhythmia information. 
Here we introduce several manifold learning techniques, the detailed theory and explanation can be accessed from the related literatures.


%%%%%%%%%%%%%%%%%%%%%%%%%%%%%%%%%%%%
%
%                     PCA
%
%%%%%%%%%%%%%%%%%%%%%%%%%%%%%%%%%%%%

\subsection{PCA}
Principal Component Analysis (PCA) is a linear dimensionality reduction method, which try to find a linear subspace of lower dimension keeping the largest variance compare to the original feature space.
PCA is by far the most popular unsupervised linear technique.
The most important task in PCA method is to find the principal components of a data set, which can be implemented by finding a linear basis with reduced dimensionality for the input data sets, with the amount of variance in the data is maximal.

Mathematically, consider data set $\{\boldsymbol{x}_i\} \subset \mathbb{R}^d$, the embedded subspace with PCA can be denote as $\{\boldsymbol{y}_i\} \subset \mathbb{R}^k$ ($d$ and $k$ are the dimension numbers). 
Then the problem of finding the ``best"  $k$ principal components can be transfer to finding $k$-dimensional subspaces that minimize the orthogonal distances.
Solve 
\begin{equation}
\label{equPCA}
\argmin \sum_{ij} (d_{ij}^2 - \Vert \bm{y}_i - \bm{y}_j \Vert^2)
%\argmin_\mathcal{S} \sum_{i=1}^d \Vert \boldsymbol{x}_i - P_\mathcal{S}(\boldsymbol{x}_i) \Vert_2^2
\end{equation}
where $P_s(\cdot)$ is the projection onto subspace $\mathcal{S}$.
Solution of this problem is to find a linear mapping $\boldsymbol{M}$ from the original space to subspace $\mathcal{S}$, which maximizes the cost function $trace(\boldsymbol{M}^Tcov(\boldsymbol{X})\boldsymbol{M}))$. Then the data low-dimensional data features $\boldsymbol{y}_i$ of data points $\boldsymbol{x}_i$ in original space can be computed by mapping $\boldsymbol{Y} = \boldsymbol{X} \boldsymbol{M}$.
A detailed theory analysis and tutorial can be found in \cite{abdi2010principal} and \cite{shlens2014tutorial} respectively. 


%%%%%%%%%%%%%%%%%%%%%%%%%%%%%%%%%%%%
%
%                 Kernel PCA
%
%%%%%%%%%%%%%%%%%%%%%%%%%%%%%%%%%%%%

\subsection{Kernel PCA}
The linear mapping could not be an accurate description of data in the nonlinear case, which happens in real world applications like the arrhythmia analysis.
In these cases, PCA will produce a large error measure. 
The geometrically nonlinear surface of data motivated different kinds of modeling approaches include kernel PCA.
Kernel PCA is a reformulation of linear PCA in a high-dimensional space constructed using kernel functions.
Kernel PCA computes the principal eigenvectors of the kernel matrix rather than the covariance matrix in the origin PCA method.
Apparently, constructing the kernel space transfer the linear based PCA into a nonlinear mapping.
The idea behind kernel PCA is to project the data into a new, higher-dimensional feature space.

Mathematically, let $n$ data points (here are the segmented heartbeat sample) $\bm{x}_i \in \mathbb{R}^d$ be given, suppose $\bm{\phi}: \mathbb{R}^d \to \mathbb{R}^D$, where $D > d$.
Assume that the mapping in the feature vectors have zero mean which is $\frac{1}{n}\sum_{i=1}^n \bm{\phi}(\bm{x})_i = 0$. Use $\bm{\Phi} = [\bm{\phi} (\bm{x}_1), \bm{\phi}(\bm{x}_2), \ldots, \bm{\phi}(\bm{x}_n)]^T \in \mathbb{R}^{n \times D}$, apply PCA to $\bm{\Phi}$.
Usually the $\bm{\phi}(\bm{x}_i)$ are unknown and it is not possible to work out the decomposition explicitly, then we define 
\begin{equation}
\kappa(\bm{x}_i, \bm{x}_i) =\bm{\phi}(\bm{x}_i)^T\bm{\phi}(\bm{x}_i)
\end{equation}
and consider $\bm{\Phi}\bm{\Phi}^T = \{\bm{\phi}(\bm{x}_i)^T \bm{\phi}(\bm{x}_i)\}$, so we have kernel matrix under the mapping of $\bm{\phi}(\cdot)$ is $\bm{K} = \{\kappa(\bm{x}_i, \bm{x}_j)\}$.
The principal $d$ eigenvectors of the centered kernel matrix can be computed, then the covariance matrix in feature space is constructed by $\kappa$.
A similar optimization problem can be summarized like in Equation \ref{equPCA} while the projection involves a kernel transformation.
Details of kernel PCA theory and application tutorial can be found in \cite{shawe2004kernel, chatfield2018introduction, van2009dimensionality}.
Kernel PCA is a kernel-based method, the mapping performed greatly relies on the choice of kernel function $\kappa$.
The linear mapping PCA could be equal to a Kernel PCA when a linear kernel is chosen. Typical kernel functions include Gaussian kernel, polynomial kernel etc. 

%%%%%%%%%%%%%%%%%%%%%%%%%%%%%%%%%%%%
%
%                     MVU
%
%%%%%%%%%%%%%%%%%%%%%%%%%%%%%%%%%%%%

\subsection{Maximum Variance Unfolding (Semidefinite Embedding)}
Since in the kernel PCA based method, the choice of kernel function is quite arbitrary. 
Sometime a poor kernel function could not lead to a good manifold embedding.
Maximum Variance Unfolding (MVU) is a technique that attempts to solve such problem by learning from data so that the kernel matrix can be obtained, which was formerly known as Semidefinite Embedding \cite{weinberger2006unsupervised}.

The notion of local isometry was proposed in MVU.
In mathematics, an isometry of a manifold is any (smooth) mapping of that manifold into itself, or into another manifold that preserves the notion of distance between points. 
Given $n$ input points $\bm{x}_i \in \mathbb{R}^d$ and a prescription for identifying neighborhood relations,  find some mapped output  $\bm{y}_i \in \mathbb{R}^k$ such that both the inputs and outputs are both locally isometric (or approximation locally isometric).
MVU starts with the construction of graph $\mathcal{G}$ illustrates the neighborhood relations. $\bm{x}_i$ is connected to its $k$ nearest neighbors, MVU tries to maximize the sum of the squared Euclidean distances between data points, with the constraint that the distances inside $\mathcal{G}$ are preserved.
Mathematically, let $\bm{y}_i$ denote the mapped representation of $\bm{x}_i$, and define a kernel matrix $\bm{K}$ as the outer product of data presentations $\bm{Y}$. After reformulating the problem turns to:
\begin{equation}
\begin{split}
& \argmax \sum_{ij}\Vert \bm{y}_i - \bm{y}_j \Vert^2  \\
& s.t. \quad  \Vert \bm{y}_i - \bm{y}_j \Vert^2 = \Vert \bm{x}_i - \bm{x}_j \Vert^2 \quad for \quad \forall (i, j) \in \mathcal{G} \\
\end{split}
\end{equation}
After we solve the semidefinite programming problem (SDP), the low-dimensional data representation $\bm{Y}$ is obtained by performing an eigenvector decomposition of $\bm{K}$.




%%%%%%%%%%%%%%%%%%%%%%%%%%%%%%%%%%%%
%
%                     ISOMAP
%
%%%%%%%%%%%%%%%%%%%%%%%%%%%%%%%%%%%%

\subsection{Isomap}
PCA finds a low-dimensional representation of the data points that best preserves the variance as which measured in the high-dimensional input space.
Later the classical multidimensional scaling (MDS) method was proposed which finds an embedding preserving the inter-point distances\cite{Kruskal1978Multidimensional}. 
The MDS method equivalent to PCA when those distances are Euclidean.
While in many case, the high-dimensional data lies on or near a curved manifold, PCA and MDS may lead a mistake in the some datasets contain essential nonlinear structures that are invisible to them, like in some face recognition dataset.
Isomap builds on classical MDS but seeks to preserve the intrinsic geometry of the data, as captured in the geodesic manifold distances between all pairs of data points \cite{tenenbaum2000isomap}. 

Mathematically, the geodesic distance between the data points  $\{\bm{x}_i\} \subset \mathbb{R}^d$ are computed so as to construct a neighborhood graph $\mathcal{G}$, where every data point $\{\bm{x}_i\}$ is connected with its $k$ nearest neighbors in the dataset.
The shortest path between two points in the graph forms an estimate of geodesic distance, which can be computed using Dijkstra's shortest-path algorithm, therefore we can get a pairwise geodesic distance matrix $\mathcal{D}$.
The low-dimensional representation can be achieved by MDS on $\mathcal{D}$.


%%%%%%%%%%%%%%%%%%%%%%%%%%%%%%%%%%%%
%
%                     LLE
%
%%%%%%%%%%%%%%%%%%%%%%%%%%%%%%%%%%%%

\subsection{Local Linear Embedding}
Local Linear Embedding is a technique that is similar to Isomap and MVU, all try to construct a graph representation of the data points.
Compare to Isomap, LLE only attempts to preserve local properties of data \cite{saul2000introduction}, which are constructed by writing the high-dimensional data points as a linear combination of their nearest neighbors.
In the low-dimensional manifold, LLE attempts to retain the reconstruction weights in the linear combinations as good as possible.



LLE describes the local properties of the manifold around a data point  $\bm{x}_i $ using a linear combination of its $k$ nearest neighbors with related  reconstruction weights $\bm{w}_i$. 
There is an assumption that the manifold is locally linear, which means that $\bm{w}_i$ of datapoints $\bm{x}_i$ are invariant to translation, rotation, and rescaling.
Then the reconstruction weights $\bm{w}_i$ can also reconstruct datapoint $\bm{y}_i$ from its neighbors in the low-dimensional data representation.
Then finding the data representation $\bm{Y}$ can be consider as an optimization problem

\begin{equation}
\begin{split}
& \argmin \sum_{i}\Vert \bm{y}_i - \sum_{j=1}^k w_{ij}\bm{y}_{i_j} \Vert^2  \\
& s.t. \quad  \Vert \bm{y}^{(k)} \Vert= 1  \quad for \quad \forall k\\
\end{split}
\end{equation}
where $\bm{y}^{(k)}$ represents the $k$-th column of the solution matrix $\bm{Y}$.



%%%%%%%%%%%%%%%%%%%%%%%%%%%%%%%%%%%%
%
%                     LE
%
%%%%%%%%%%%%%%%%%%%%%%%%%%%%%%%%%%%%
\subsection{Laplacian Eigenmaps}
Similar to LLE, the Laplacian Eigenmaps (LE) methods find low-dimensional data representations by preserving manifold's local properties based on pairwise distances of near neighbors.
LE compute distances between data points with its neighbors using weights based on the rule: the first nearest neighbor contributes more to the cost function than the distance between the datapoint and its second nearest neighbor.
The minimization of the cost function in LE is defined as an eigen-problem in spectral graph theory.
 
Consider a neighborbood graph $\mathcal{G}$ in which each data sample $\bm{x}_i$ is connected to its $k$ nearest neighbors.
The weights of the edges are computed using Gaussian kernel function:
\begin{equation}
w_{ij} = e^{-\frac{\Vert \bm{x}_i - \bm{x}_j \Vert^2}{2\sigma^2}}
\end{equation}
where $\sigma$ indicates the Gaussian variance.
Then we can get a sparse adjacency matrix $\bm{W}$, and the optimization problem:
\begin{equation}
\argmin \sum_{ij} \Vert\bm{y}_i - \bm{y}_j \Vert^2 w_{ij}
\end{equation}
Large weights $w_{ij}$ correspond to small distances of $\bm{x}_i$ and $\bm{x}_j$, then the difference between their manifold representation $\bm{y}_i$ and $\bm{x}_j$ highly contributes to the cost function.
So the nearby points in the original space (high-dimensional) are put as close as possible in the low-dimensional manifold, which forms a better representation for input datasets \cite{belkin2003laplacian}.

%%%%%%%%%%%%%%%%%%%%%%%%%%%%%%%%%%%%
%
%                     SAE
%
%%%%%%%%%%%%%%%%%%%%%%%%%%%%%%%%%%%%
\subsection{Autoencoder}
While the above methods are based on some geometrical explantations, Auto Encoders are a family of nonlinear dimensional reduction methods.
Autoencoders are neural networks had been used and explained in the term deep learning where had been used to predict the input by let the output is trained to be as similar as possible as the input.
The structure of autoencoders are illustrated in Figure \ref{figauto}, on the end of autoencoder there were less hidden nodes than input nodes by encoding as much information as it can be for the hidden neural nodes. 
Superficially high-dimensional and complex phenomena can be mapped into some lower dimensional manifold, performs good representation of inputs.
 
\begin{figure}[!t]
\centering
\label{figauto}
\includegraphics[width=3.5in]{autoencoder.pdf}
\caption{A Multi-layer Autoencoder Reconstruction Structure}
\label{fig_sim}
\end{figure}

Suppose the input $\mathcal{X} = \{\bm{\hat{x}}_i\}$ and the learned autoencoder based manifold $\mathcal{Y} = \{\bm{y}_i\}$. The autoencoder structure is constructed with an $a$-layer neural network in which the output $h_{W, b}(\bm{x}) $ equals to the input $\bm{x} = (\bm{\hat{x}}_1, \bm{\hat{x}}_2, \ldots, \bm{\hat{x}}_n)^T$. Let an autoencoder parameters are $\bm{W}$ and $\bm{b}$, and $f, g$ were encoder and decoder respectively. 
The output of autoencoder is:
\begin{equation}
h_{\bm{W}, \bm{b}}(\bm{\hat{x}}) = g(f(\bm{\hat{x}})) \approx \bm{\hat{x}}
\end{equation}
Then the training of the autoencoder can be consider as a problem of:

\begin{equation}
\argmin_{\bm{W}, \bm{b}} \frac{1}{2} \Vert h_{\bm{W}, \bm{b}}(\bm{\hat{x}}) - \bm{\hat{x}} \Vert^2
\end{equation}
This multi-layer neural network structure can be trained using the backpropagation method in a unsupervised way. In deep learning literatures a layer-wised method had been widely used as \cite{Bengio2009deeplearning} illustrated.

Once the architecture parameters of $\bm{W}, \bm{b}$ are learned, one can use the last layer of encoder as the output. From the view point of manifold learning, the autoencoder mapped the original input into a manifold feature space. One can use the manifold feature for further classification problem.

%%%%%%%%%%%%%%%%%%%%%%%%%%%%%%%%%%%%
%
%                     Classifier
%
%%%%%%%%%%%%%%%%%%%%%%%%%%%%%%%%%%%%

\subsection{Arrhythmia Classification with Softmax Classifier}
After the manifold embedding procedure for the input $\mathcal{X} = \{\bm{x}_i\}$, we get the data features  $\mathcal{Y} = \{\bm{y}_i\}$ on manifold, with a cascade classifier the arrhythmia classification could be accomplished.

The softmax function is a generalization of the logistic function which had been widely used as a muli-class classifier or the last layer of neural networks \cite{christopher2006pattern}.
In probability theory, the outputs can represent categorical distribution over $k$ possible outcomes.
We use softmax regression as the classifier using the manifold representations as the training samples for the classifier, and then deal the test dataset using the same embedding methods.
The whole task can be illustrated as two phases, the learning phase and the predict(classifying) phase.
In the learning phase, first the ECG samples are embedded into some manifold to get correspond features with dimensional reduction methods as we illustrated in Section II.A to Section II.G.
After the embedding, the achieved  features are used for training a softmax based classifier in a supervised learning method. 
In the prediction phase use the same strategy to get embedded features, and the class label for arrhythmia analysis can be achieved by the trained softmax classifier.
 
Mathematically, let $\{\bm{c}_i\}$ illustrated the class label, $\mathcal{X} = \{\bm{x}_i\}$ as input data, and  $\mathcal{Y} = \{\bm{y}_i\}$ the manifold feature.
Let $1\{\cdot\}$be the indicator function, so that $\bm{1}\{true \quad statement\} = 1$, and $\bm{1}\{false \quad statement\} = 0$. 
Then the cost function for the softmax regression classifier between the ground truth label $c^{(i)}$ and the estimated label $g(\bm{y}^{(i)}, \bm{\theta})$ is computed by loss function L:
\begin{equation}
J(\bm{\theta}) = \frac{1}{n}\sum_{i=1}^nL(c^{(i)}, g(\bm{y}^{(i)}, \bm{\theta})) + \lambda r(\bm{\theta})
\end{equation}
where the loss function of $\bm{y}^{(i)}$ is defined as:
\begin{equation}
L(c^{(i)}, g(\bm{y}^{(i)}, \bm{\theta})) = -\sum_{k=1}^K\bm{1}\{c^{(i)} = k\} \log \frac{e^{\bm{\theta}_k^T\bm{y}^{(i)}}}{\sum_{c=1}^Ke^{\bm{\theta}_k^T\bm{y}^{(i)}} }
\end{equation}
while the regularization term we use is
\begin{equation}
r(\bm{\theta}) = \frac{1}{2} \sum_{k=1}^K\sum_{j=1}^d \bm{\theta}_{kj}^2
\end{equation}
It is well-known that softmax regression is a convex model which can find a global optimal solution \cite{ren2017robust}, then the training for the classifier is:
\begin{equation}
\bm{\theta} = \argmin J(\bm{\theta})
\end{equation}
The class label output of the softmax classifier can be achieved via:
\begin{equation}
c_i = \argmax g(\bm{y}^{(i)}, \bm{\theta}) 
\end{equation}

%%%%%%%%%%%%%%%%%%%%%%%%%%%%%%%%%%%%
%%%%%%%%%%%%%%%%%%%%%%%%%%%%%%%%%%%%
%
%                     Experiments and Results
%
%%%%%%%%%%%%%%%%%%%%%%%%%%%%%%%%%%%%
%%%%%%%%%%%%%%%%%%%%%%%%%%%%%%%%%%%%


\section{Materials And Experiments}
We include the above mentioned methods for the ECG arrhythmia classification task for a comparative study for an illustration of manifold learning in time series based biomedical signal analysis.
In this section we first make a short description about the task and data types since this problem had been quite well discussed in related literatures, after which the experiment flows are illustrated.



%%%%%%%%%%%%%%%%%%%%%%%%%%%%%%%%%%%%
%
%                     ECG Arrhythmia \& Heartbeat Classification
%
%%%%%%%%%%%%%%%%%%%%%%%%%%%%%%%%%%%%
%While the autoencoder based method 
\subsection{ECG Arrhythmia \& Heartbeat Classification}


\begin{table*}[!htbp]
\begin{center}
\begin{threeparttable}
\caption{AAMI Classes Mapped from MIT-BIH ECG Databases}
\label{table1}
\begin{tabular}{cllllll}
\hline
& AAMI heartbeat classes & N & S & V & F & Q \\
& Description  &Any heartbeat not in & Supraventricular ectopic  & Ventricular ectopic  & Fusion beat & Unknown beat \\
&                     &the S,V,F or Q class & beat   		     & beat	      &	     &          \\
\hline
& MIT-BIH heartbeat  &Normal beat (1)            & Aberrated atrial premature & Ventricular escape & Fusion of ventricular& Paced beat (12)\\
&  types (codes)   &  					  &  beat (11)		    &  beat(10)	            & \& normal beat (6)		      &             \\

&                     &Left bundle branch   &  Nodal (junctional) &Premature ventricular& 	 &  Unclassifiable \\
&                     &block beat (2)            & premature beat (7)    &contraction (5)         &	   & beat (13)   \\

&                     & Right bundle branch & Atrial premature & Ventricular flutter &    &  Fusion of paced  \\
&                     & block beat (3) & contraction (8)     & wave (31) & 	   & and normal (38)  \\

&                     & Nodal escape  & Premature or ectopic & 	     &  &                          \\
&                     & beat (11)	  & supraventricular beat(9) &     &   &     \\

&                     & Atrial escape beat (34)   &  	  				      & 				            & 			      &                          \\
%\hline
%& MITBIH-AR$^a$(100,687) & 89,925$^b$   & 2,774   & 7,171   & 802    & 15        \\
%& MITBIH-LT(667,347) & 600,197  &  150  & 64,090  & 	2,906   & 0       \\

\hline
\end{tabular}
%\begin{tablenotes}

%\end{tablenotes}
\end{threeparttable}

\end{center}
     \end{table*}


After the segmentation for the ECG records, we got plenty of ECG waveform samples with variety categories. 
Since different physiological disorder may be reflected on the different type of abnormal heartbeat rhythms. 
For the task of classification, it is quite important to determine the classes would be used. 
In the early literatures, there were no unified class labels for an ECG classification problem. 
The MIT-BIH Arrhythmia Database was the first available set of standard test material for evaluation of arrhythmia detectors; it played an important role in stimulating manufacturers of arrhythmia analyzers to compete by objectively measurable performance. 
The annotations in the open database for the ECG categories adopted the ANSI/AAMI EC57: 1998/(R)2008 standard AAMI (2008), which recommended to group the heartbeats into five classes: on-ectopic beats (N); supraventricular ectopic beat (S); ventricular ectopic beat (V); fusion of a V and a N (F); unknown beat type (Q). These classes or labels have been widely used in the ECG classification tasks. 
The normal beat, supraventricular ectopic beat and the ventricular ectopic beat categories were used much more frequently while the unknown beat type were abandoned because of its clinical valueless.
Table \ref{table1} illustrated the categories for the arrhythmia analysis task.

The most popular dataset used in arrhythmia analysis task is the MIT-BIH Arrhythmia Database [23] which contains 48 half-hour recordings each containing two 30-min ECG lead signals (lead I and lead II), sampled at 340Hz. 
However, only the lead I was used in this study. 
In agreement with the AAMI recommended practice, the four recordings with paced beats were removed from the analysis. 

Besides, a unlabelled data set of ambulatory electrocardiography collected from the Sun Yat-sen Cardiovascular Hospital were used in this study, which includes $24$-hour recordings of $100$ subjects with partial arrhythmia ECG samples. 
This dataset would be used in the autoencoder pre-training, which has been quite popular in the deep learning literatures.
Even though we explore autoencoder method from a manifold angle, a pre-training procedure is powerful strategy in extracting manifold representations for arrhythmia analysis.




%%%%%%%%%%%%%%%%%%%%%%%%%%%%%%%%%%%%
%
%                     Data Preprocessing
%
%%%%%%%%%%%%%%%%%%%%%%%%%%%%%%%%%%%%
\subsection{Data Preprocessing}
In the preprocessing stage, filtering algorithms were adapted to remove the artifact signals from the ECG signal. 
The signals include baseline wander, power line interference, and high-frequency noise. 
The segmentation and R wave detection algorithms had been explored in \cite{afonso} and segmentation program of Laguna \cite{sornmo2006electrocardiogram} was adapted, which also had been validated by other related work \cite{chaza}. 
After segmentation, the data samples become $340$-dimensional time series.



%%%%%%%%%%%%%%%%%%%%%%%%%%%%%%%%%%%%
%
%                     Intrinsic Dimension Analysis
%
%%%%%%%%%%%%%%%%%%%%%%%%%%%%%%%%%%%%
\subsection{Intrinsic Dimension Analysis}
The dimension of the embedding is a key parameter for manifold feature achieving.  
If the dimension is too small, important data features are "collapsed", meanwhile if the dimension is too large, the projection into a manifold is too large, leading to a noisy or unstable data feature.
When performing the manifold learning methods, the learning parameter settings could be quite arbitrary sometimes.
The intrinsic dimension analysis could help for the parameter determination.
Especially in the neural network based autoencoder, there is an insufficiency of method for determining the dimension for output layer.
Intrinsic dimension analysis provide a reasonable method for parameter determination.

The widely adopted method of intrinsic dimension analysis include correlation dimension estimation, nearest neighbor dimension estimation\cite{costa2005estimating}, geodesic minimum spanning tree\cite{costa2004geodesic}, packing numbers \cite{kegl2003intrinsic}, maximum likelihood estimation method \cite{levina2005maximum}, etc.
Here we use the the maximum likelihood estimation method for intrinsic dimension analysis.
The dimension from the input dimension turns to $14$ with the maximum likelihood estimation based intrinsic dimension analysis.



%%%%%%%%%%%%%%%%%%%%%%%%%%%%%%%%%%%%
%
%                     Sample Imbalance Treatment
%
%%%%%%%%%%%%%%%%%%%%%%%%%%%%%%%%%%%%
\subsection{Sample Imbalance Treatment}
For the arrhythmia analysis, a typical problem from the viewpoint of data analysis is the unbalancedness of data.
It is due to the characters from the cardiac physiological activity.
Even for some severe heart disease, the ECG achieved would include much more normal beats than the arrhythmia beats after segmentation.
The skewed distribution makes manifold learning algorithms less effective, especially in the prediction of minority class samples.
Usually, a confusion matrix had been build for the performance assess to prove the validity of proposed methods.
Here we arbitrarily delete part of the data sample to relieve the imbalance problem, and use the confusion matrix to evaluate the classification performance. 
The class distributions are illustrated in Table \ref{distriofData}, a more complicate and detailed analysis of impact for imbalance impact can be found in \cite{wang2012multiclass} and more about two-class situation in \cite{he2009learning}.



\begin{table}[!htbp]
\begin{center}
\begin{threeparttable}
\caption{Comparisons Of Imbalanced And Balanced Of DS-I From MITBHI-AR}
\label{distriofData}
\begin{tabular}{ccccccc}
\hline
& Classes & N & S & V  & F \\
\hline
& Imbalanced  &90431  & 2774  & 7698   &8023\\
& Ratio & 83.0\%&  2.5\%& 7.1\%  & 7.3\% \\
\hline
& Balanced  & 18086 & 2774  & 7698 & 8023         \\
& Ratio & 49.5\%&  7.6\%& 21\%  & 21.9\% \\
\hline
\end{tabular}
\end{threeparttable}
\end{center}
\end{table}
     

%%%%%%%%%%%%%%%%%%%%%%%%%%%%%%%%%%%%
%
%                
%
%%%%%%%%%%%%%%%%%%%%%%%%%%%%%%%%%%%%     
     
     
\subsection{Manifold Learning and Classification}
The first step for the arrhythmia classification problem is mapping the data into some manifold to gain related data embedding. It can be considered as some feature extraction, or representation learning in the machine learning literatures.
The difference was that these manifold based methods are inspired from some geometrical concepts.
For the mentioned methods, the PCA, kernel based PCA, MVU, Isomap, LLE, Laplacian Eigenmaps only involve the MIT-BIH-AR data samples for geometrical transformation, in the autoencoder based manifold learning, we use some unlabelled data to train a multilayer neural network in a stacked manner.

As Algorithm \ref{alg1} illustrated, let $\mathcal{M}$ illustrated some mapping method with rule $\mathcal{R}$, and the related softmax classifier as $\mathcal{S}$, consider the balanced dataset as DS1 (from the MIT-BHI AR database). 

\begin{algorithm}
 \caption{Manifold Learning With Rule $\mathcal{R}$}
 \label{alg1}
 \begin{algorithmic}
 \REQUIRE Data Set I (DS-I) Samples: $ \mathcal{X} = \{\bm{x}_1, \bm{x}_2, \ldots, \bm{x}_n\}$;
 \ENSURE Sample Classes: $\mathcal{C} = \{c_1, c_2, \ldots, c_3\}$
 \STATE $TrainSet, TestSet \leftarrow \mathcal{X}$
 \STATE $\mathcal{M} \leftarrow $ Process $TrainSet$ with criteria $\mathcal{R}$
 \STATE $\mathcal{Y} \leftarrow$ Map $TrainSet$ with $\mathcal{M}$ 
 \STATE $\mathcal{S} \leftarrow$ Train the Softmax classifier with $\mathcal{Y}$
 \STATE $\mathcal{C} \leftarrow \mathcal{S}, \mathcal{M}(TestSet)$
  \end{algorithmic}
 \end{algorithm}


As Algorithm \ref{alg2} illustrated, consider the multilayer autoencoder method, the random initialization of the neural network parameter might fail with the limited data samples in DS-I. Then we first use the unlabelled data set DS-II with the same preprocessing method to pre-train the neural network, it had been widely used in deep learning method \cite{vincent2010stacked, goodfellow2016deep}. 
After the pre-training state, we use samples to do further training for the representation, and the cascade classification.
A $5$-fold cross validation had been applied for the whole training and predict, refer \cite{kohavi1995study} for detail.


\begin{algorithm}
 \caption{Manifold Learning With Autoencoder}
 \label{alg2}
 \begin{algorithmic}
 \REQUIRE Data Set I (DS-I) Samples: $ \mathcal{X} = \{\bm{x}_1, \bm{x}_2, \ldots, \bm{x}_n\}$; \\$\quad \quad $Data Set II (DS-II) Samples: $\hat{\mathcal{X}} = \{\bm{\hat{x}}_1, \bm{\hat{x}}_2, \ldots, \bm{\hat{x}}_n\}$
 \ENSURE Sample Classes: $\mathcal{C} = \{c_1, c_2, \ldots, c_n\}$
 \STATE $\bm{\theta} \leftarrow $ self-reconstruction of $\mathcal{\hat{X}}$
  \STATE $TrainSet, TestSet \leftarrow \mathcal{X}$
  \STATE $\mathcal{Y} \leftarrow$ Map $TrainSet$ with encoder parameters $\bm{\theta}$ 
 \STATE $\mathcal{S} \leftarrow$ Train the Softmax classifier with $\mathcal{Y}$
 \STATE $\mathcal{C} \leftarrow \mathcal{S}, \bm{\theta}, TestSet$
  \end{algorithmic}
 \end{algorithm}

 


%%%%%%%%%%%%%%%%%%%%%%%%%%%%%%%%%%%%
%
%       Performance Analysis              
%
%%%%%%%%%%%%%%%%%%%%%%%%%%%%%%%%%%%%

\subsection{Performance Analysis}
Correctly detected episodes are termed true positive (TPs) episodes, while the missed as false negatives (FNs).
When a negative episode in some class is predicted as true is called true negative (TNs), and a false positive (FPs) if it is considered as positive. And the sensitivity(Se), specificity(Sp), and F1-score are defined as following with respectively:

\begin{equation}
\begin{split}
Se = \frac{TP}{TP+FN} \quad \\
 Sp = \frac{TN}{TN+FP}
\end{split}
\end{equation}
Refer \cite{mar} for an further knowledge for parameters for algorithm evaluation in ECG data analysis.



%%%%%%%%%%%%%%%%%%%%%%%%%%%%%%%%%%%%
%
%                     Results and Discussion
%
%%%%%%%%%%%%%%%%%%%%%%%%%%%%%%%%%%%%


\section{Results and Discussion}
Using the projectors from Section II, we use the time series based manifold feature to perform classification task. 
From the intrinsic dimensional analysis, we have the parameter for the dimension of 1-lead ECG heartbeat is $14$.
Then we use it as a key parameter in the embedding calculations. 

\subsection{Experimental Results With Manifold Representations}

%%%%%%%%%%%%%%%%%%%%%%%%%%%%%%%%%%%%
%                     PCA
%%%%%%%%%%%%%%%%%%%%%%%%%%%%%%%%%%%%

In Table \ref{pca} the results for using PCA dimensionality reduction with softmax classifier are illustrated, the input dimension of $340$ was mapped into a $14$-dimensional feature space.
\begin{table}[!htbp]
\begin{center}
\begin{threeparttable}
\caption{Test Result PCA With Softmax}
\label{pca}
\begin{tabular}{ccccccc}
\hline
\multicolumn{5}{r}{Algorithm label} \\
\cline{3-6}
		&  & N & S      & V    & F       & T\\
\hline
 Reference & N & 3397 &  23  &  244   & 26    &  3617 \\
	label  & S &  105    & 395  &   45   & 9    &  555\\
		   & V &  246    & 34    & 1194 & 58    &  1540\\
		   & F &  71   & 10    & 90    & 1432   &  1605\\		
\hline
\end{tabular}
\end{threeparttable}
\end{center}
\end{table}

%%%%%%%%%%%%%%%%%%%%%%%%%%%%%%%%%%%%
%                     KPCA
%%%%%%%%%%%%%%%%%%%%%%%%%%%%%%%%%%%%

In Table \ref{kpca} the results for using Gaussian kernel based PCA was used, the problem faced for kernel PCA is the high computation consumption for kernel matrix  computation, the size of the kernel matrix is proportional to the square of the number of samples in the dataset.

\begin{table}[!htbp]
\begin{center}
\begin{threeparttable}
\caption{Test Result Gaussian Kernel With Softmax}
\label{kpca}
\begin{tabular}{ccccccc}
\hline
\multicolumn{5}{r}{Algorithm label} \\
\cline{3-6}
		&  & N & S      & V    & F       & T\\
\hline
 Reference & N & 3421 &  30  &  132  & 34    &  3617 \\
	label  & S & 113     & 401  &  34   & 7    &  555\\
		   & V & 238  & 29   & 1203 & 70    &  1540\\
		   & F & 67    & 8    & 74    & 1456   &  1605\\		
\hline
\end{tabular}
\end{threeparttable}
\end{center}
\end{table}


%%%%%%%%%%%%%%%%%%%%%%%%%%%%%%%%%%%%
%                     MVU
%%%%%%%%%%%%%%%%%%%%%%%%%%%%%%%%%%%%
Table \ref{mvu} and Table \ref{isomap} illustrate the results for using maximum variance unfolding and Isomap respectively. Since they both use geodesic distances between data points as constraints, the face the sample problem in topological instability, this is because they may construct erroneous connections in the neighborhood graph. 
\begin{table}[!htbp]
\begin{center}
\begin{threeparttable}
\caption{Test Result MVU With Softmax}
\label{mvu}
\begin{tabular}{ccccccc}
\hline
\multicolumn{5}{r}{Algorithm label} \\
\cline{3-6}
		&  & N & S      & V    & F       & T\\
\hline
 Reference & N & 3452 &  20  &  138   & 7   &  3617 \\
	label  & S &  98    & 357  &  48   & 52   &  555\\
		   & V &  231    & 27    & 1183 & 99    &  1540\\
		   & F &  51    & 12    & 11    & 1531   &  1605\\		
\hline
\end{tabular}
\end{threeparttable}
\end{center}
\end{table}


%%%%%%%%%%%%%%%%%%%%%%%%%%%%%%%%%%%%
%                     ISOMAP
%%%%%%%%%%%%%%%%%%%%%%%%%%%%%%%%%%%%



\begin{table}[!htbp]
\begin{center}
\begin{threeparttable}
\caption{Test Result Isomap With Softmax}
\label{isomap}
\begin{tabular}{ccccccc}
\hline
\multicolumn{5}{r}{Algorithm label} \\
\cline{3-6}
		&  & N & S      & V    & F       & T\\
\hline
 Reference & N & 3349 &  39  &  192   & 37   &  3617 \\
	label  & S &  95    & 387  &   61   & 12    &  555\\
		   & V &  194    & 27    & 1257 & 62    &  1540\\
		   & F &  68    & 19    & 157    & 1361  &  1605\\		
\hline
\end{tabular}

\end{threeparttable}
\end{center}
\end{table}


%%%%%%%%%%%%%%%%%%%%%%%%%%%%%%%%%%%%
%                     LLE
%%%%%%%%%%%%%%%%%%%%%%%%%%%%%%%%%%%%


In Table \ref{LLE} the results for using local linear embedding for classification were shown. They preserve local properties of the data, and shows 

\begin{table}[!htbp]
\begin{center}
\begin{threeparttable}
\caption{Test Result LLE With Softmax}
\label{LLE}
\begin{tabular}{ccccccc}
\hline
\multicolumn{5}{r}{Algorithm label} \\
\cline{3-6}
		&  & N & S      & V    & F       & T\\
\hline
 Reference & N & 3206 &  41 &  287   & 83    &  3617 \\
	label  & S &  98    & 349  &  93   & 15    &  555\\
		   & V &  197    & 43    &  1267 & 33    &  1540\\
		   & F &  101    & 34    & 68   & 1402   &  1605\\		
\hline
\end{tabular}

\end{threeparttable}
\end{center}
\end{table}


%%%%%%%%%%%%%%%%%%%%%%%%%%%%%%%%%%%%
%                     LE
%%%%%%%%%%%%%%%%%%%%%%%%%%%%%%%%%%%%

In Table \ref{Laplacian} the results for using laplacian eigenmaps  embedding for classification were shown:


\begin{table}[!htbp]
\begin{center}
\begin{threeparttable}
\caption{Test Result for LE With Softmax}
\label{Laplacian}
\begin{tabular}{ccccccc}
\hline
\multicolumn{5}{r}{Algorithm label} \\
\cline{3-6}
		&  & N & S      & V    & F       & T\\
\hline
 Reference & N & 3476 & 19  &  103 & 19    &  3617 \\
	label  & S &  23    & 421  &   80   & 31    &  555\\
		   & V &  191    & 21    & 1269 & 59    &  1540\\
		   & F &  80   & 14    & 44    & 1467   &  1605\\		
\hline
\end{tabular}

\end{threeparttable}
\end{center}
\end{table}


%%%%%%%%%%%%%%%%%%%%%%%%%%%%%%%%%%%%
%                     SAE
%%%%%%%%%%%%%%%%%%%%%%%%%%%%%%%%%%%%

In Table \ref{Laplacian} the results for using multi-layer autoencoder embedding for classification were shown:


\begin{table}[!htbp]
\begin{center}
\begin{threeparttable}
\caption{Test Result For Multi-Layer Autoencoder Network With Softmax}
\label{table7}
\begin{tabular}{ccccccc}
\hline
\multicolumn{5}{r}{Algorithm label} \\
\cline{3-6}
		&  & N & S      & V    & F       & T\\
\hline
 Reference & N & 3578 & 5 &  24   & 10    &  3617 \\
	label  & S &  26    &504  &  20   & 5    &  555\\
		   & V &  19    & 39    & 1471 & 11    &  1540\\
		   & F &  37    & 87    & 29    & 1452   &  1605\\		
\hline
\end{tabular}
\end{threeparttable}
\end{center}
\end{table}






  
\subsection{Comparison of different multi-Layer autoencoder based methods}
The autoencoder based nonlinear dimensionality reduction methods possess powerful representation ability.
While the choice for the structure node numbers could be arbitrary sometime, here we arbitrarily tested several networks settings, and make a comparison with their experiment results. We found that for arrhythmia analysis problems, we may not need too ``deep" structures. The results are shown in Table \ref{aeresults}.

  
\begin{table}[!htbp]
\begin{center}
\begin{threeparttable}
\caption{Performance Comparison For Different Multi-Layer Autoencoders}
\label{aeresults}
\begin{tabular}{cccccccc}
\hline
&  Items & N & S & V & F   \\
\hline
&3-Layer  &99.32\% & 83.21\% &98.24\% &  87.72\%  \\
&4-Layer  &99.13\% & 74.23\%  & 97.12\% & 65.23\%       \\
&5-Layer &98.74\% & 72.87\% & 96.98\% &  58.57\% \\
\hline
\end{tabular}
\end{threeparttable}
\end{center}
\end{table}
  
   

\subsection{Comparison with other methods}
In \cite{mar} a feature selection strategy had been used for the ECG classification, a multi-layer perceptron method had been used to assess the mode, they try to use the artificial extracted features from the time series to build a automated model for ambulatory monitoring and arrhythmia analysis, typical results from \cite{chaza, melgan, jiang} are also illustrated for results comparison.

\begin{table}[!htbp]
\begin{center}
\begin{threeparttable}
\caption{Performance Comparison For Manifold Learning Methods}
\label{table1}
\begin{tabular}{cccccccc}
\hline
&  Items & N & S & V & F   \\
\hline

&Proposed &$\bm{98.30}$\% & $\bm{84.71}$\% & $\bm{95.27}$ \%& $\bm{87.72}$\%\\
& Mar. \cite{mar}  &84.85\% & 82.29\% & 86.72\% &51.55\%      \\


& Chazal. \cite{chaza} &86.86\% & 83.83\%  & 77.74\% & 89.43\%      \\

& Melgani.\cite{melgan}  &89.12\% & -  & 89.97\%  & -       \\

& Jiang.\cite{jiang}  & 94.51\% & 50.59\%  & 86.61\% & 35.78\%       \\


\hline
\end{tabular}
\end{threeparttable}
\end{center}
\end{table}
     
     


%%%%%%%%%%%%%%%%%%%%%%%%%%%%%%%%%%%%
%
%                     Conclusion and Future Work
%
%%%%%%%%%%%%%%%%%%%%%%%%%%%%%%%%%%%%

\section{Conclusion and Future Work}
In this paper, we investigated the manifold learning methods to facilitate the multi-class classification problem of electrocardiography in arrhythmia analysis.
Typical manifold learning methods had been investigated, capable of generating meaningful features for classification.
The manifold methods always keep some constraints based on geometrical condition.
PCA preserve the data variance, kernel PCA focuses on retaining the pairwise distances, MVU has the constraints in maximizing the Euclidean distance between data points with keeping the distances inside the neighborhood graph, Isomap attempting to preserve pairwise geodesic distances, LLE only consider the local properties of data i.e. locally linearity, and for Laplacian Eigenmaps works in a ameliorate way, a weighted distance in neighborhood had been used.

All these methods may face some weak points under the geometrical constraints, compare to the multi-layer autoencoder neural networks.
The multi-layer autoencoder learn a neural network structure by training with lots of data in a self-mapping-and-reconstruction way.
Though it is not easy to employ some graceful theoretical explanations, its powerful representation ability could perform much better in this classification task.

The other important concern in the arrhythmia analysis problem is the multi-lead combinational analysis, since in different signal channels the heart physiological information was represented in different way. The manifold learning methods could help to reduction the redundancy dimension, and propose better feature for analysis.


% An example of a floating figure using the graphicx package.
% Note that \label must occur AFTER (or within) \caption.
% For figures, \caption should occur after the \includegraphics.
% Note that IEEEtran v1.7 and later has special internal code that
% is designed to preserve the operation of \label within \caption
% even when the captionsoff option is in effect. However, because
% of issues like this, it may be the safest practice to put all your
% \label just after \caption rather than within \caption{}.
%
% Reminder: the "draftcls" or "draftclsnofoot", not "draft", class
% option should be used if it is desired that the figures are to be
% displayed while in draft mode.
%


% Note that the IEEE typically puts floats only at the top, even when this
% results in a large percentage of a column being occupied by floats.


% An example of a double column floating figure using two subfigures.
% (The subfig.sty package must be loaded for this to work.)
% The subfigure \label commands are set within each subfloat command,
% and the \label for the overall figure must come after \caption.
% \hfil is used as a separator to get equal spacing.
% Watch out that the combined width of all the subfigures on a 
% line do not exceed the text width or a line break will occur.
%
%\begin{figure*}[!t]
%\centering
%\subfloat[Case I]{\includegraphics[width=2.5in]{box}%
%\label{fig_first_case}}
%\hfil
%\subfloat[Case II]{\includegraphics[width=2.5in]{box}%
%\label{fig_second_case}}
%\caption{Simulation results for the network.}
%\label{fig_sim}
%\end{figure*}
%
% Note that often IEEE papers with subfigures do not employ subfigure
% captions (using the optional argument to \subfloat[]), but instead will
% reference/describe all of them (a), (b), etc., within the main caption.
% Be aware that for subfig.sty to generate the (a), (b), etc., subfigure
% labels, the optional argument to \subfloat must be present. If a
% subcaption is not desired, just leave its contents blank,
% e.g., \subfloat[].


% An example of a floating table. Note that, for IEEE style tables, the
% \caption command should come BEFORE the table and, given that table
% captions serve much like titles, are usually capitalized except for words
% such as a, an, and, as, at, but, by, for, in, nor, of, on, or, the, to
% and up, which are usually not capitalized unless they are the first or
% last word of the caption. Table text will default to \footnotesize as
% the IEEE normally uses this smaller font for tables.
% The \label must come after \caption as always.
%
%\begin{table}[!t]
%% increase table row spacing, adjust to taste
%\renewcommand{\arraystretch}{1.3}
% if using array.sty, it might be a good idea to tweak the value of
% \extrarowheight as needed to properly center the text within the cells
%\caption{An Example of a Table}
%\label{table_example}
%\centering
%% Some packages, such as MDW tools, offer better commands for making tables
%% than the plain LaTeX2e tabular which is used here.
%\begin{tabular}{|c||c|}
%\hline
%One & Two\\
%\hline
%Three & Four\\
%\hline
%\end{tabular}
%\end{table}


% Note that the IEEE does not put floats in the very first column
% - or typically anywhere on the first page for that matter. Also,
% in-text middle ("here") positioning is typically not used, but it
% is allowed and encouraged for Computer Society conferences (but
% not Computer Society journals). Most IEEE journals/conferences use
% top floats exclusively. 
% Note that, LaTeX2e, unlike IEEE journals/conferences, places
% footnotes above bottom floats. This can be corrected via the
% \fnbelowfloat command of the stfloats package.





% if have a single appendix:
%\appendix[Proof of the Zonklar Equations]
% or
%\appendix  % for no appendix heading
% do not use \section anymore after \appendix, only \section*
% is possibly needed

% use appendices with more than one appendix
% then use \section to start each appendix
% you must declare a \section before using any
% \subsection or using \label (\appendices by itself
% starts a section numbered zero.)
%


\appendices
% you can choose not to have a title for an appendix
% if you want by leaving the argument blank

% use section* for acknowledgment
\section*{Acknowledgment}
%The author would like to thank for the reviewer for advices.
This work was supported by the National Natural Science Foundation of China, Grant No. 71531004.


% Can use something like this to put references on a page
% by themselves when using endfloat and the captionsoff option.
\ifCLASSOPTIONcaptionsoff
  \newpage
\fi



% trigger a \newpage just before the given reference
% number - used to balance the columns on the last page
% adjust value as needed - may need to be readjusted if
% the document is modified later
%\IEEEtriggeratref{8}
% The "triggered" command can be changed if desired:
%\IEEEtriggercmd{\enlargethispage{-5in}}

% references section

% can use a bibliography generated by BibTeX as a .bbl file
% BibTeX documentation can be easily obtained at:
% http://mirror.ctan.org/biblio/bibtex/contrib/doc/
% The IEEEtran BibTeX style support page is at:
% http://www.michaelshell.org/tex/ieeetran/bibtex/
\bibliographystyle{IEEEtran}
% argument is your BibTeX string definitions and bibliography database(s)
\bibliography{ecgManifold}

% <OR> manually copy in the resultant .bbl file
% set second argument of \begin to the number of references
% (used to reserve space for the reference number labels box)
%\begin{thebibliography}{1}

%\bibitem{IEEEhowto:kopka}
%H.~Kopka and P.~W. Daly, \emph{A Guide to \LaTeX}, 3rd~ed.\hskip 1em plus
 % 0.5em minus 0.4em\relax Harlow, England: Addison-Wesley, 1999.

%\end{thebibliography}

% biography section
% 
% If you have an EPS/PDF photo (graphicx package needed) extra braces are
% needed around the contents of the optional argument to biography to prevent
% the LaTeX parser from getting confused when it sees the complicated
% \includegraphics command within an optional argument. (You could create
% your own custom macro containing the \includegraphics command to make things
% simpler here.)
%\begin{IEEEbiography}[{\includegraphics[width=1in,height=1.25in,clip,keepaspectratio]{mshell}}]{Michael Shell}
% or if you just want to reserve a space for a photo:

%\begin{IEEEbiography}[{\includegraphics[width=1in,height=1.25in,clip,keepaspectratio]{YY.pdf}}]{Yan Yan}
%(M'15) received the B.Eng. degree and the MSc in Instrument Engineering at the Harbin Institute of Technology in 2010 and 2012 respectively. He worked as research assistance at the Shenzhen Institutes of Advanced Technology, Chinese Academy of Sciences from 2012 to 2014. He is working on his PhD in Computer Science started from 2014.His research interests are digital signal processing, machine learning, and control system.
%\end{IEEEbiography}

%\begin{IEEEbiography}[{\includegraphics[width=1in,height=1.25in,clip,keepaspectratio]{WL.pdf}}]{Lei Wang}
%received the B.Eng. degree in information and control engineering, and the Ph.D. degree in biomedical engineering from Xi’an Jiaotong University, Xi’an, China, in 1995 and 2000, respectively. He was with the University of Glasgow, Glasgow, U.K., and Imperial College London, London, U.K., from 2000 to 2008. He is currently a full professor with the Shenzhen Institutes of Advanced Technology, Chinese Academy of Sciences, Shenzhen, China. He has published over 200 scientific papers, authored four book chapters, and holds 60 patents. His current research interests include body sensor network, digital signal processing, and biomedical engineering
%\end{IEEEbiography}

% if you will not have a photo at all:

\begin{IEEEbiographynophoto}{Yan Yan}
received the B.Eng. degree and the MSc in Instrument Engineering at the Harbin Institute of Technology in 2010 and 2012 respectively. He worked as research assistance at the Shenzhen Institutes of Advanced Technology, Chinese Academy of Sciences from 2012 to 2014. He is working on his PhD in Computer Science started from 2014.His research interests are digital signal processing, machine learning, and control system.
\end{IEEEbiographynophoto}

\begin{IEEEbiographynophoto}{Lei Wang}
received the B.Eng. degree in information and control engineering, and the Ph.D. degree in biomedical engineering from Xi’an Jiaotong University, Xi’an, China, in 1995 and 2000, respectively. He was with the University of Glasgow, Glasgow, U.K., and Imperial College London, London, U.K., from 2000 to 2008. He is currently a full professor with the Shenzhen Institutes of Advanced Technology, Chinese Academy of Sciences, Shenzhen, China. He has published over 200 scientific papers, authored four book chapters, and holds 60 patents. His current research interests include body sensor network, digital signal processing, and biomedical engineering
\end{IEEEbiographynophoto}


% insert where needed to balance the two columns on the last page with
% biographies
%\newpage

% You can push biographies down or up by placing
% a \vfill before or after them. The appropriate
% use of \vfill depends on what kind of text is
% on the last page and whether or not the columns
% are being equalized.

%\vfill

% Can be used to pull up biographies so that the bottom of the last one
% is flush with the other column.
%\enlargethispage{-5in}



% that's all folks
\end{document}


