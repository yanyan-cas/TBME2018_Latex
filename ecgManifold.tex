
%% bare_jrnl.tex
%% V1.4b
%% 2015/08/26
%% by Michael Shell
%% see http://www.michaelshell.org/
%% for current contact information.
%%
%% This is a skeleton file demonstrating the use of IEEEtran.cls
%% (requires IEEEtran.cls version 1.8b or later) with an IEEE
%% journal paper.
%%
%% Support sites:
%% http://www.michaelshell.org/tex/ieeetran/
%% http://www.ctan.org/pkg/ieeetran
%% and
%% http://www.ieee.org/

%%*************************************************************************
%% Legal Notice:
%% This code is offered as-is without any warranty either expressed or
%% implied; without even the implied warranty of MERCHANTABILITY or
%% FITNESS FOR A PARTICULAR PURPOSE! 
%% User assumes all risk.
%% In no event shall the IEEE or any contributor to this code be liable for
%% any damages or losses, including, but not limited to, incidental,
%% consequential, or any other damages, resulting from the use or misuse
%% of any information contained here.
%%
%% All comments are the opinions of their respective authors and are not
%% necessarily endorsed by the IEEE.
%%
%% This work is distributed under the LaTeX Project Public License (LPPL)
%% ( http://www.latex-project.org/ ) version 1.3, and may be freely used,
%% distributed and modified. A copy of the LPPL, version 1.3, is included
%% in the base LaTeX documentation of all distributions of LaTeX released
%% 2003/12/01 or later.
%% Retain all contribution notices and credits.
%% ** Modified files should be clearly indicated as such, including  **
%% ** renaming them and changing author support contact information. **
%%*************************************************************************


% *** Authors should verify (and, if needed, correct) their LaTeX system  ***
% *** with the testflow diagnostic prior to trusting their LaTeX platform ***
% *** with production work. The IEEE's font choices and paper sizes can   ***
% *** trigger bugs that do not appear when using other class files.       ***                          ***
% The testflow support page is at:
% http://www.michaelshell.org/tex/testflow/



\documentclass[journal]{IEEEtran}
%
% If IEEEtran.cls has not been installed into the LaTeX system files,
% manually specify the path to it like:
% \documentclass[journal]{../sty/IEEEtran}




\usepackage{graphicx}
\usepackage{amsfonts}
\usepackage{amsmath} % assumes amsmath package installed
\DeclareMathOperator*{\argmax}{argmax}
\DeclareMathOperator*{\argmin}{argmin}
\usepackage{bm}

\usepackage{algorithmic}
\usepackage{algorithm}
\renewcommand{\algorithmicrequire}{\textbf{Input:}}
 \renewcommand{\algorithmicensure}{\textbf{Output:}}

\usepackage{lipsum}
% Some very useful LaTeX packages include:
% (uncomment the ones you want to load)


% *** MISC UTILITY PACKAGES ***
%
%\usepackage{ifpdf}
% Heiko Oberdiek's ifpdf.sty is very useful if you need conditional
% compilation based on whether the output is pdf or dvi.
% usage:
% \ifpdf
%   % pdf code
% \else
%   % dvi code
% \fi
% The latest version of ifpdf.sty can be obtained from:
% http://www.ctan.org/pkg/ifpdf
% Also, note that IEEEtran.cls V1.7 and later provides a builtin
% \ifCLASSINFOpdf conditional that works the same way.
% When switching from latex to pdflatex and vice-versa, the compiler may
% have to be run twice to clear warning/error messages.






% *** CITATION PACKAGES ***
%
%\usepackage{cite}
% cite.sty was written by Donald Arseneau
% V1.6 and later of IEEEtran pre-defines the format of the cite.sty package
% \cite{} output to follow that of the IEEE. Loading the cite package will
% result in citation numbers being automatically sorted and properly
% "compressed/ranged". e.g., [1], [9], [2], [7], [5], [6] without using
% cite.sty will become [1], [2], [5]--[7], [9] using cite.sty. cite.sty's
% \cite will automatically add leading space, if needed. Use cite.sty's
% noadjust option (cite.sty V3.8 and later) if you want to turn this off
% such as if a citation ever needs to be enclosed in parenthesis.
% cite.sty is already installed on most LaTeX systems. Be sure and use
% version 5.0 (2009-03-20) and later if using hyperref.sty.
% The latest version can be obtained at:
% http://www.ctan.org/pkg/cite
% The documentation is contained in the cite.sty file itself.






% *** GRAPHICS RELATED PACKAGES ***
%
\ifCLASSINFOpdf
  % \usepackage[pdftex]{graphicx}
  % declare the path(s) where your graphic files are
  % \graphicspath{{../pdf/}{../jpeg/}}
  % and their extensions so you won't have to specify these with
  % every instance of \includegraphics
  % \DeclareGraphicsExtensions{.pdf,.jpeg,.png}
\else
  % or other class option (dvipsone, dvipdf, if not using dvips). graphicx
  % will default to the driver specified in the system graphics.cfg if no
  % driver is specified.
  % \usepackage[dvips]{graphicx}
  % declare the path(s) where your graphic files are
  % \graphicspath{{../eps/}}
  % and their extensions so you won't have to specify these with
  % every instance of \includegraphics
  % \DeclareGraphicsExtensions{.eps}
\fi
% graphicx was written by David Carlisle and Sebastian Rahtz. It is
% required if you want graphics, photos, etc. graphicx.sty is already
% installed on most LaTeX systems. The latest version and documentation
% can be obtained at: 
% http://www.ctan.org/pkg/graphicx
% Another good source of documentation is "Using Imported Graphics in
% LaTeX2e" by Keith Reckdahl which can be found at:
% http://www.ctan.org/pkg/epslatex
%
% latex, and pdflatex in dvi mode, support graphics in encapsulated
% postscript (.eps) format. pdflatex in pdf mode supports graphics
% in .pdf, .jpeg, .png and .mps (metapost) formats. Users should ensure
% that all non-photo figures use a vector format (.eps, .pdf, .mps) and
% not a bitmapped formats (.jpeg, .png). The IEEE frowns on bitmapped formats
% which can result in "jaggedy"/blurry rendering of lines and letters as
% well as large increases in file sizes.
%
% You can find documentation about the pdfTeX application at:
% http://www.tug.org/applications/pdftex





% *** MATH PACKAGES ***
%
%\usepackage{amsmath}
% A popular package from the American Mathematical Society that provides
% many useful and powerful commands for dealing with mathematics.
%
% Note that the amsmath package sets \interdisplaylinepenalty to 10000
% thus preventing page breaks from occurring within multiline equations. Use:
%\interdisplaylinepenalty=2500
% after loading amsmath to restore such page breaks as IEEEtran.cls normally
% does. amsmath.sty is already installed on most LaTeX systems. The latest
% version and documentation can be obtained at:
% http://www.ctan.org/pkg/amsmath





% *** SPECIALIZED LIST PACKAGES ***
%
%\usepackage{algorithmic}
% algorithmic.sty was written by Peter Williams and Rogerio Brito.
% This package provides an algorithmic environment fo describing algorithms.
% You can use the algorithmic environment in-text or within a figure
% environment to provide for a floating algorithm. Do NOT use the algorithm
% floating environment provided by algorithm.sty (by the same authors) or
% algorithm2e.sty (by Christophe Fiorio) as the IEEE does not use dedicated
% algorithm float types and packages that provide these will not provide
% correct IEEE style captions. The latest version and documentation of
% algorithmic.sty can be obtained at:
% http://www.ctan.org/pkg/algorithms
% Also of interest may be the (relatively newer and more customizable)
% algorithmicx.sty package by Szasz Janos:
% http://www.ctan.org/pkg/algorithmicx




% *** ALIGNMENT PACKAGES ***
%
%\usepackage{array}
% Frank Mittelbach's and David Carlisle's array.sty patches and improves
% the standard LaTeX2e array and tabular environments to provide better
% appearance and additional user controls. As the default LaTeX2e table
% generation code is lacking to the point of almost being broken with
% respect to the quality of the end results, all users are strongly
% advised to use an enhanced (at the very least that provided by array.sty)
% set of table tools. array.sty is already installed on most systems. The
% latest version and documentation can be obtained at:
% http://www.ctan.org/pkg/array


% IEEEtran contains the IEEEeqnarray family of commands that can be used to
% generate multiline equations as well as matrices, tables, etc., of high
% quality.




% *** SUBFIGURE PACKAGES ***
%\ifCLASSOPTIONcompsoc
%  \usepackage[caption=false,font=normalsize,labelfont=sf,textfont=sf]{subfig}
%\else
%  \usepackage[caption=false,font=footnotesize]{subfig}
%\fi
% subfig.sty, written by Steven Douglas Cochran, is the modern replacement
% for subfigure.sty, the latter of which is no longer maintained and is
% incompatible with some LaTeX packages including fixltx2e. However,
% subfig.sty requires and automatically loads Axel Sommerfeldt's caption.sty
% which will override IEEEtran.cls' handling of captions and this will result
% in non-IEEE style figure/table captions. To prevent this problem, be sure
% and invoke subfig.sty's "caption=false" package option (available since
% subfig.sty version 1.3, 2005/06/28) as this is will preserve IEEEtran.cls
% handling of captions.
% Note that the Computer Society format requires a larger sans serif font
% than the serif footnote size font used in traditional IEEE formatting
% and thus the need to invoke different subfig.sty package options depending
% on whether compsoc mode has been enabled.
%
% The latest version and documentation of subfig.sty can be obtained at:
% http://www.ctan.org/pkg/subfig




% *** FLOAT PACKAGES ***
%
%\usepackage{fixltx2e}
% fixltx2e, the successor to the earlier fix2col.sty, was written by
% Frank Mittelbach and David Carlisle. This package corrects a few problems
% in the LaTeX2e kernel, the most notable of which is that in current
% LaTeX2e releases, the ordering of single and double column floats is not
% guaranteed to be preserved. Thus, an unpatched LaTeX2e can allow a
% single column figure to be placed prior to an earlier double column
% figure.
% Be aware that LaTeX2e kernels dated 2015 and later have fixltx2e.sty's
% corrections already built into the system in which case a warning will
% be issued if an attempt is made to load fixltx2e.sty as it is no longer
% needed.
% The latest version and documentation can be found at:
% http://www.ctan.org/pkg/fixltx2e


%\usepackage{stfloats}
% stfloats.sty was written by Sigitas Tolusis. This package gives LaTeX2e
% the ability to do double column floats at the bottom of the page as well
% as the top. (e.g., "\begin{figure*}[!b]" is not normally possible in
% LaTeX2e). It also provides a command:
%\fnbelowfloat
% to enable the placement of footnotes below bottom floats (the standard
% LaTeX2e kernel puts them above bottom floats). This is an invasive package
% which rewrites many portions of the LaTeX2e float routines. It may not work
% with other packages that modify the LaTeX2e float routines. The latest
% version and documentation can be obtained at:
% http://www.ctan.org/pkg/stfloats
% Do not use the stfloats baselinefloat ability as the IEEE does not allow
% \baselineskip to stretch. Authors submitting work to the IEEE should note
% that the IEEE rarely uses double column equations and that authors should try
% to avoid such use. Do not be tempted to use the cuted.sty or midfloat.sty
% packages (also by Sigitas Tolusis) as the IEEE does not format its papers in
% such ways.
% Do not attempt to use stfloats with fixltx2e as they are incompatible.
% Instead, use Morten Hogholm'a dblfloatfix which combines the features
% of both fixltx2e and stfloats:
%
% \usepackage{dblfloatfix}
% The latest version can be found at:
% http://www.ctan.org/pkg/dblfloatfix




%\ifCLASSOPTIONcaptionsoff
%  \usepackage[nomarkers]{endfloat}
% \let\MYoriglatexcaption\caption
% \renewcommand{\caption}[2][\relax]{\MYoriglatexcaption[#2]{#2}}
%\fi
% endfloat.sty was written by James Darrell McCauley, Jeff Goldberg and 
% Axel Sommerfeldt. This package may be useful when used in conjunction with 
% IEEEtran.cls'  captionsoff option. Some IEEE journals/societies require that
% submissions have lists of figures/tables at the end of the paper and that
% figures/tables without any captions are placed on a page by themselves at
% the end of the document. If needed, the draftcls IEEEtran class option or
% \CLASSINPUTbaselinestretch interface can be used to increase the line
% spacing as well. Be sure and use the nomarkers option of endfloat to
% prevent endfloat from "marking" where the figures would have been placed
% in the text. The two hack lines of code above are a slight modification of
% that suggested by in the endfloat docs (section 8.4.1) to ensure that
% the full captions always appear in the list of figures/tables - even if
% the user used the short optional argument of \caption[]{}.
% IEEE papers do not typically make use of \caption[]'s optional argument,
% so this should not be an issue. A similar trick can be used to disable
% captions of packages such as subfig.sty that lack options to turn off
% the subcaptions:
% For subfig.sty:
% \let\MYorigsubfloat\subfloat
% \renewcommand{\subfloat}[2][\relax]{\MYorigsubfloat[]{#2}}
% However, the above trick will not work if both optional arguments of
% the \subfloat command are used. Furthermore, there needs to be a
% description of each subfigure *somewhere* and endfloat does not add
% subfigure captions to its list of figures. Thus, the best approach is to
% avoid the use of subfigure captions (many IEEE journals avoid them anyway)
% and instead reference/explain all the subfigures within the main caption.
% The latest version of endfloat.sty and its documentation can obtained at:
% http://www.ctan.org/pkg/endfloat
%
% The IEEEtran \ifCLASSOPTIONcaptionsoff conditional can also be used
% later in the document, say, to conditionally put the References on a 
% page by themselves.




% *** PDF, URL AND HYPERLINK PACKAGES ***
%
%\usepackage{url}
% url.sty was written by Donald Arseneau. It provides better support for
% handling and breaking URLs. url.sty is already installed on most LaTeX
% systems. The latest version and documentation can be obtained at:
% http://www.ctan.org/pkg/url
% Basically, \url{my_url_here}.




% *** Do not adjust lengths that control margins, column widths, etc. ***
% *** Do not use packages that alter fonts (such as pslatex).         ***
% There should be no need to do such things with IEEEtran.cls V1.6 and later.
% (Unless specifically asked to do so by the journal or conference you plan
% to submit to, of course. )


% correct bad hyphenation here
%\hyphenation{op-tical net-works semi-conduc-tor}


\begin{document}
%
% paper title
% Titles are generally capitalized except for words such as a, an, and, as,
% at, but, by, for, in, nor, of, on, or, the, to and up, which are usually
% not capitalized unless they are the first or last word of the title.
% Linebreaks \\ can be used within to get better formatting as desired.
% Do not put math or special symbols in the title.
\title{Exploring Manifold Learning Methods For Automated Arrhythmia Analysis}

%
%
% author names and IEEE memberships
% note positions of commas and nonbreaking spaces ( ~ ) LaTeX will not break
% a structure at a ~ so this keeps an author's name from being broken across
% two lines.
% use \thanks{} to gain access to the first footnote area
% a separate \thanks must be used for each paragraph as LaTeX2e's \thanks
% was not built to handle multiple paragraphs
%

\author{Yan~Yan,~\IEEEmembership{Student Member,~IEEE,}
       % x~x,~\IEEEmembership{Senior Member,~IEEE,}
        and~Lei~Wang,~\IEEEmembership{}% <-this % stops a space
\thanks{Y. Yan is with the Shenzhen Institutes of Advanced Technology, Chinese Academy of Sciences, and University of Chinese Academy of Sciences, China, (e-mail: yan.yan@siat.ac.cn).}% <-this % stops a space
\thanks{L. Wang is with Shenzhen Institutes of Advanced Technology, Chinese Academy of Sciences.}% <-this % stops a space
\thanks{Manuscript received ; revised. }}

% note the % following the last \IEEEmembership and also \thanks - 
% these prevent an unwanted space from occurring between the last author name
% and the end of the author line. i.e., if you had this:
% 
% \author{....lastname \thanks{...} \thanks{...} }
%                     ^------------^------------^----Do not want these spaces!
%
% a space would be appended to the last name and could cause every name on that
% line to be shifted left slightly. This is one of those "LaTeX things". For
% instance, "\textbf{A} \textbf{B}" will typeset as "A B" not "AB". To get
% "AB" then you have to do: "\textbf{A}\textbf{B}"
% \thanks is no different in this regard, so shield the last } of each \thanks
% that ends a line with a % and do not let a space in before the next \thanks.
% Spaces after \IEEEmembership other than the last one are OK (and needed) as
% you are supposed to have spaces between the names. For what it is worth,
% this is a minor point as most people would not even notice if the said evil
% space somehow managed to creep in.



% The paper headers
\markboth{}%
{xx \MakeLowercase{\textit{et al.}}: Exploring Manifold Learning Methods For Automated Arrhythmia Analysis}
% The only time the second header will appear is for the odd numbered pages
% after the title page when using the two side option.
% 
% *** Note that you probably will NOT want to include the author's ***
% *** name in the headers of peer review papers.                   ***
% You can use \ifCLASSOPTIONpeerreview for conditional compilation here if
% you desire.




% If you want to put a publisher's ID mark on the page you can do it like
% this:
%\IEEEpubid{0000--0000/00\$00.00~\copyright~2015 IEEE}
% Remember, if you use this you must call \IEEEpubidadjcol in the second
% column for its text to clear the IEEEpubid mark.



% use for special paper notices
%\IEEEspecialpapernotice{(Invited Paper)}




% make the title area
\maketitle

% As a general rule, do not put math, special symbols or citations
% in the abstract or keywords.
\begin{abstract}
\lipsum[1]
\lipsum[1]
\end{abstract}

% Note that keywords are not normally used for peerreview papers.
\begin{IEEEkeywords}
IEEE, IEEEtran, journal, \LaTeX, paper, template.
\end{IEEEkeywords}






% For peer review papers, you can put extra information on the cover
% page as needed:
% \ifCLASSOPTIONpeerreview
% \begin{center} \bfseries EDICS Category: 3-BBND \end{center}
% \fi
%
% For peerreview papers, this IEEEtran command inserts a page break and
% creates the second title. It will be ignored for other modes.
\IEEEpeerreviewmaketitle



\section{Introduction}
% The very first letter is a 2 line initial drop letter followed
% by the rest of the first word in caps.
% 
% form to use if the first word consists of a single letter:
% \IEEEPARstart{A}{demo} file is ....
% 
% form to use if you need the single drop letter followed by
% normal text (unknown if ever used by the IEEE):
% \IEEEPARstart{A}{}demo file is ....
% 
% Some journals put the first two words in caps:
% \IEEEPARstart{T}{his demo} file is ....
% 
% Here we have the typical use of a "T" for an initial drop letter
% and "HIS" in caps to complete the first word.
\IEEEPARstart{T}{he} computer aided system for pathology detection and diagnosis are based on a feature space constructed from abundant clinical attributes.
The noninvasive, inexpensive and well-established technology of electrocardiographic signal in mobile health or personal health has the greatest popularity in heart function analysis. 
Automated arrhythmia analysis provides indispensable assist in long-term clinical monitoring, and a large number of approaches have been proposed for the task, easing the diagnosis of arrhythmic changes as well as further inspection, e.g., heart rate variability or heart turbulence analysis.
Information in  electrocardiography signals related to the cardiac electrical activity is therefore represented by a large dimensional space from the data analysis angle.
Plenty of parameters were extracted from the high dimensional data as the indicators for diagnosis. 
The large dimensional space hinders a proper interpretation of the embedded symbolic physiology in the feature space \cite{delgado2009dimensionality}.
However, the increasingly developing of the artificial intelligence domain and machine learning methods provides powerful tools to deal with the large dimensional electrocardiography data.





Lots of algorithms had been proposed for the classification and detection of electrocardiography signals. 
The electrocardiography classification or detection task had been divided into two parts: the feature extraction process and classifier. 
Simple classifier such as linear discriminants \cite{chaza} and kNN \cite{melgan}, more complex classifiers like neural networks \cite{jiang, olmez, lin, osowski}, fuzzy inference engines \cite{osowski, kundu}, hidden Markov model \cite{andreao, coast}, independent component analysis \cite{zhu} and support vector machine  \cite{melgan, kampoura, khandoker} were also adopted by lots of researchers.  



Beyond the classifier, the performance of a recognition system highly depends on the determination of extracted electrocardiography features. Time domain features, frequency domain features, and statistical measures features for six fundamental waves (PQRSTU) had been used in feature extraction process \cite{chia}. 
Time domain features like morphological features include shapes, amplitudes, and durations were adapted primarily in \cite{jekova, christove, can}, frequency domain features like wavelet transformation were widely used \cite{inan}, \cite{banerjee} stationary features like higher-order statistics also had been developed. 
Principal component analysis \cite{stam} and Hermite functions \cite{lager} have been used in electrocardiography classification and related analysis technologies as well.
Almost every single published paper proposes a new set of features to be used, or a new combination of the existing ones \cite{mar}.


The results from these algorithms or models were not amenable to expert labelling, as well as for the identification of complex relationships between subjects and clinical conditions \cite{clifford}.
But for the ambulatory electrocardiography clinical application, as well as the normal application in daily healthcare monitoring for cardiac function or early warning of heart disease, an automated algorithm or model would have significant meaning.
The application of artificial intelligence methods has become an important trend in electrocardiography for the recognition and classification of different arrhythmia types \cite{clifford}. 
The data explosion puts forward the new request to the method of data processing and information mining.



Over the past decades computational techniques proliferated in the pattern recognition field, simultaneously the applications in electrocardiography recognition, detection and classification for relevant trends, patterns, and outliers. 
Most of the literatures in the electrocardiography classification task were focused on the supervised learning methods, as in unsupervised learning methods were infrequently used, which needs a lot of effort in labelling data. The MIT-BIH database \cite{physionet} was the most widely used data in the classification and detection algorithm developments, while mass unlabelled electrocardiography data had been ignored due to the supervise learning approaches essential. 
Unsupervised learning methods become crucial in mining or analysing unlabelled data, as the unlabelled electrocardiography data accumulated. Unsupervised learning-based approaches and the application to electrocardiogram classification in literatures mainly include clustering-based techniques \cite{lager, nishizawa, maier}, self-adaptive neural network-based methods \cite{palreddy, risk} and some hybrid unsupervised learning systems \cite{tadejko}. 

Manifold learning is a kind of method attempting to uncover the manifold structure in a data set, it can also be considered as a kind of nonlinear dimensional reduction technique.
It has been widely used in multivariate data classification, data analysis, and data visualization.
Manifold learning directly map the input data space into some feature space to find a better descriptor to analysis the real world problems. 




\lipsum[1-3]
%%%%%%%%%%%%%%%%%%%%%%%%%%%%%%%%%%%%%%%%%%%%%%%%%%%%%%%%%%%%%%
%
%                     SECTION III: Methodology
%
%%%%%%%%%%%%%%%%%%%%%%%%%%%%%%%%%%%%%%%%%%%%%%%%%%%%%%%%%%%%%%

\section{Methodology}
The arrhythmia analysis problem can be generally considered as one time series based classification problem from the viewpoint of data analysis.
Consider a set of $n$ electrocardiography based time series data set $\boldsymbol{X} = \{\boldsymbol{x}_i\} \subset \mathbb{R}^d$, is given as the samples.
Heartbeats in $\boldsymbol{X}$ include both labelled and unlabelled samples, the task is to estimate the labels of unlabelled data with some methods based on the labelled and relative methods.
Here we adopt several manifold learning based techniques to embed the high dimensional input data into some lower dimension subspace. 
With the techniques some data properties were preserved in the embedded subspace (feature space).
Then we train a classifier in the feature space using the labelled data, and use the classifier to classify the unlabelled data. From which we can get the arrhythmia information. 
Here we introduce several manifold learning techniques, the detailed theory and explanation can be accessed from the related literatures.


%%%%%%%%%%%%%%%%%%%%%%%%%%%%%%%%%%%%
%
%                     PCA
%
%%%%%%%%%%%%%%%%%%%%%%%%%%%%%%%%%%%%

\subsection{PCA}
Principal Component Analysis (PCA) is a linear dimensionality reduction method, which try to find a linear subspace of lower dimension keeping the largest variance compare to the original feature space.
PCA is by far the most popular unsupervised linear technique.
The most important task in PCA method is to find the principal components of a data set, which can be implemented by finding a linear basis with reduced dimensionality for the input data sets, with the amount of variance in the data is maximal.

Mathematically, consider data set $\{\boldsymbol{x}_i\} \subset \mathbb{R}^d$, the embedded subspace with PCA can be denote as $\{\boldsymbol{y}_i\} \subset \mathbb{R}^k$ ($d$ and $k$ are the dimension numbers). 
Then the problem of finding the ``best"  $k$ principal components can be transfer to finding $k$-dimensional subspaces that minimize the orthogonal distances.
Solve 
\begin{equation}
\label{equPCA}
\argmin_\mathcal{S} \sum_{i=1}^d \Vert \boldsymbol{x}_i - P_\mathcal{S}(\boldsymbol{x}_i) \Vert_2^2
\end{equation}
where $P_s(\cdot)$ is the projection onto subspace $\mathcal{S}$.
Solution of this problem is to find a linear mapping $\boldsymbol{M}$ from the original space to subspace $\mathcal{S}$, which maximizes the cost function $trace(\boldsymbol{M}^Tcov(\boldsymbol{X})\boldsymbol{M}))$. Then the data low-dimensional data features $\boldsymbol{y}_i$ of data points $\boldsymbol{x}_i$ in original space can be computed by mapping $\boldsymbol{Y} = \boldsymbol{X} \boldsymbol{M}$.
A detailed theory analysis and tutorial can be found in \cite{abdi2010principal} and \cite{shlens2014tutorial} respectively. 


%%%%%%%%%%%%%%%%%%%%%%%%%%%%%%%%%%%%
%
%                 Kernel PCA
%
%%%%%%%%%%%%%%%%%%%%%%%%%%%%%%%%%%%%

\subsection{Kernel PCA}
The linear mapping could not be an accurate description of data in the nonlinear case, which happens in real world applications like the arrhythmia analysis.
In these cases, PCA will produce a large error measure. 
The geometrically nonlinear surface of data motivated different kinds of modeling approaches include kernel PCA.
Kernel PCA is a reformulation of linear PCA in a high-dimensional space constructed using kernel functions.
Kernel PCA computes the principal eigenvectors of the kernel matrix rather than the covariance matrix in the origin PCA method.
Apparently, constructing the kernel space transfer the linear based PCA into a nonlinear mapping.
The idea behind kernel PCA is to project the data into a new, higher-dimensional feature space.

Mathematically, let $n$ data points (here are the segmented heartbeat sample) $\bm{x}_i \in \mathbb{R}^d$ be given, suppose $\bm{\phi}: \mathbb{R}^d \to \mathbb{R}^D$, where $D > d$.
Assume that the mapping in the feature vectors have zero mean which is $\frac{1}{n}\sum_{i=1}^n \bm{\phi}(\bm{x})_i = 0$. Use $\bm{\Phi} = [\bm{\phi} (\bm{x}_1), \bm{\phi}(\bm{x}_2), \ldots, \bm{\phi}(\bm{x}_n)]^T \in \mathbb{R}^{n \times D}$, apply PCA to $\bm{\Phi}$.
Usually the $\bm{\phi}(\bm{x}_i)$ are unknown and it is not possible to work out the decomposition explicitly, then we define 
\begin{equation}
\kappa(\bm{x}_i, \bm{x}_i) =\bm{\phi}(\bm{x}_i)^T\bm{\phi}(\bm{x}_i)
\end{equation}
and consider $\bm{\Phi}\bm{\Phi}^T = \{\bm{\phi}(\bm{x}_i)^T \bm{\phi}(\bm{x}_i)\}$, so we have kernel matrix under the mapping of $\bm{\phi}(\cdot)$ is $\bm{K} = \{\kappa(\bm{x}_i, \bm{x}_j)\}$.
The principal $d$ eigenvectors of the centered kernel matrix can be computed, then the covariance matrix in feature space is constructed by $\kappa$.
A similar optimization problem can be summarized like in Equation \ref{equPCA} while the projection involves a kernel transformation.
Details of kernel PCA theory and application tutorial can be found in \cite{shawe2004kernel, chatfield2018introduction, van2009dimensionality}.
Kernel PCA is a kernel-based method, the mapping performed greatly relies on the choice of kernel function $\kappa$.
The linear mapping PCA could be equal to a Kernel PCA when a linear kernel is chosen. Typical kernel functions include Gaussian kernel, polynomial kernel etc. 

%%%%%%%%%%%%%%%%%%%%%%%%%%%%%%%%%%%%
%
%                     MVU
%
%%%%%%%%%%%%%%%%%%%%%%%%%%%%%%%%%%%%

\subsection{Maximum Variance Unfolding (Semidefinite Embedding)}
Since in the kernel PCA based method, the choice of kernel function is quite arbitrary. 
Sometime a poor kernel function could not lead to a good manifold embedding.
Maximum Variance Unfolding (MVU) is a technique that attempts to solve such problem by learning from data so that the kernel matrix can be obtained, which was formerly known as Semidefinite Embedding \cite{weinberger2006unsupervised}.

The notion of local isometry was proposed in MVU.
In mathematics, an isometry of a manifold is any (smooth) mapping of that manifold into itself, or into another manifold that preserves the notion of distance between points. 
Given $n$ input points $\bm{x}_i \in \mathbb{R}^d$ and a prescription for identifying neighborhood relations,  find some mapped output  $\bm{y}_i \in \mathbb{R}^k$ such that both the inputs and outputs are both locally isometric (or approximation locally isometric).
MVU starts with the construction of graph $\mathcal{G}$ illustrates the neighborhood relations. $\bm{x}_i$ is connected to its $k$ nearest neighbors, MVU tries to maximize the sum of the squared Euclidean distances between data points, with the constraint that the distances inside $\mathcal{G}$ are preserved.
Mathematically, let $\bm{y}_i$ denote the mapped representation of $\bm{x}_i$, and define a kernel matrix $\bm{K}$ as the outer product of data presentations $\bm{Y}$. After reformulating the problem turns to:
\begin{equation}
\begin{split}
& \argmax \sum_{ij}\Vert \bm{y}_i - \bm{y}_j \Vert^2  \\
& s.t. \quad  \Vert \bm{y}_i - \bm{y}_j \Vert^2 = \Vert \bm{x}_i - \bm{x}_j \Vert^2 \quad for \quad \forall (i, j) \in \mathcal{G} \\
\end{split}
\end{equation}
After we solve the semidefinite programming problem (SDP), the low-dimensional data representation $\bm{Y}$ is obtained by performing an eigenvector decomposition of $\bm{K}$.




%%%%%%%%%%%%%%%%%%%%%%%%%%%%%%%%%%%%
%
%                     ISOMAP
%
%%%%%%%%%%%%%%%%%%%%%%%%%%%%%%%%%%%%

\subsection{Isomap}
PCA finds a low-dimensional representation of the data points that best preserves the variance as which measured in the high-dimensional input space.
Later the classical multidimensional scaling (MDS) method was proposed which finds an embedding preserving the inter-point distances\cite{Kruskal1978Multidimensional}. 
The MDS method equivalent to PCA when those distances are Euclidean.
While in many case, the high-dimensional data lies on or near a curved manifold, PCA and MDS may lead a mistake in the some datasets contain essential nonlinear structures that are invisible to them, like in some face recognition dataset.
Isomap builds on classical MDS but seeks to preserve the intrinsic geometry of the data, as captured in the geodesic manifold distances between all pairs of data points \cite{tenenbaum2000isomap}. 

Mathematically, the geodesic distance between the data points  $\{\bm{x}_i\} \subset \mathbb{R}^d$ are computed so as to construct a neighborhood graph $\mathcal{G}$, where every data point $\{\bm{x}_i\}$ is connected with its $k$ nearest neighbors in the dataset.
The shortest path between two points in the graph forms an estimate of geodesic distance, which can be computed using Dijkstra's shortest-path algorithm, therefore we can get a pairwise geodesic distance matrix $\mathcal{D}$.
The low-dimensional representation can be achieved by MDS on $\mathcal{D}$.


%%%%%%%%%%%%%%%%%%%%%%%%%%%%%%%%%%%%
%
%                     LLE
%
%%%%%%%%%%%%%%%%%%%%%%%%%%%%%%%%%%%%

\subsection{Local Linear Embedding}
Local Linear Embedding is a technique that is similar to Isomap and MVU, all try to construct a graph representation of the data points.
Compare to Isomap, LLE only attempts to preserve local properties of data \cite{saul2000introduction}, which are constructed by writing the high-dimensional data points as a linear combination of their nearest neighbors.
In the low-dimensional manifold, LLE attempts to retain the reconstruction weights in the linear combinations as good as possible.



LLE describes the local properties of the manifold around a data point  $\bm{x}_i $ using a linear combination of its $k$ nearest neighbors with related  reconstruction weights $\bm{w}_i$. 
There is an assumption that the manifold is locally linear, which means that $\bm{w}_i$ of datapoints $\bm{x}_i$ are invariant to translation, rotation, and rescaling.
Then the reconstruction weights $\bm{w}_i$ can also reconstruct datapoint $\bm{y}_i$ from its neighbors in the low-dimensional data representation.
Then finding the data representation $\bm{Y}$ can be consider as an optimization problem

\begin{equation}
\begin{split}
& \argmin \sum_{i}\Vert \bm{y}_i - \sum_{j=1}^k w_{ij}\bm{y}_{i_j} \Vert^2  \\
& s.t. \quad  \Vert \bm{y}^{(k)} = 1 \quad for \quad \forall k\\
\end{split}
\end{equation}



%%%%%%%%%%%%%%%%%%%%%%%%%%%%%%%%%%%%
%
%                     LE
%
%%%%%%%%%%%%%%%%%%%%%%%%%%%%%%%%%%%%
\subsection{Laplacian Eigenmaps}
\cite{belkin2003laplacian}






%%%%%%%%%%%%%%%%%%%%%%%%%%%%%%%%%%%%
%
%                     Sammon
%
%%%%%%%%%%%%%%%%%%%%%%%%%%%%%%%%%%%%
\subsection{Sammon Mapping}






%%%%%%%%%%%%%%%%%%%%%%%%%%%%%%%%%%%%
%
%                     SAE
%
%%%%%%%%%%%%%%%%%%%%%%%%%%%%%%%%%%%%
\subsection{Multilayer Autoencoder}




%%%%%%%%%%%%%%%%%%%%%%%%%%%%%%%%%%%%
%
%                     Classifier
%
%%%%%%%%%%%%%%%%%%%%%%%%%%%%%%%%%%%%

\subsection{Classifier Adopted}








\section{Material And Experiments}
\subsection{Data Collection}
As for the signal acquiring process, different kinds of sample rates might be involved, for common ECG acquisition device the sample rate would be 128Hz, 250Hz, 340Hz or 500Hz even higher. 
In murine studies, a sampling rate as high as 2kHz is considered sufficiently.
Arbitrary resizing would be an ideal procedure to handle with the different sampling rate from a different data source to build the datasets for analysis, which would be adopted in the experiment to keep data consistency.


\subsection{ECG Arrhythmia}
After the segmentation for the ECG records, we got plenty of ECG waveform sam- ples with variety categories. Since different physiological disorder may be reflected on the different type of abnormal heartbeat rhythms. For the task of classification, it is quite important to determine the classes would be used. In the early liter- ature, there were no unified class labels for an ECG classification problem. The MIT-BIH Arrhythmia Database was the first available set of standard test material for evaluation of arrhythmia detectors; it played an important role in stimulating manufacturers of arrhythmia analyzers to compete by objectively measurable per- formance. The annotations in the open database for the ECG categories adopted the ANSI/AAMI EC57: 1998/(R)2008 standard AAMI (2008), which recommended to group the heartbeats into five classes: on-ectopic beats (N as the Figure ?? (a)); supraventricular ectopic beat (S as the Figure ?? (b)); ventricular ectopic beat (V as the Figure ?? (c) and (d)); fusion of a V and a N (F); unknown beat type (Q). These classes or labels have been widely used in the ECG classification tasks re- lated literature (Table ?? illustrated). The normal beat, supraventricular ectopic beat and the ventricular ectopic beat categories were used much more frequently while the unknown beat type were abandoned because of its clinical valueless.


\subsection{Data Source}
Data from the ambulatory electrocardiography database were used in this study, which includes recordings of 100 subjects with arrhythmia along with normal sinus rhythm. The database contains 100 recordings, each containing a 3- lead 24-hour long electrocardiography which were bandpass filtered at 0.1-100Hz and sampled at 128Hz. In this study, only the lead I data were adapted after preprocessing in the classification task. The reference average heart beats for each sample has 97,855 beats for the 24-hour long recording, and the reference arrhythmia average is 1,810 beats which were estimated by a commercial software (this statistics aim to indicate the existence for arrhythmia samples, which should not be consider as a experiment preset).
The MIT-BIH Arrhythmia Database [23] contains 48 half- hour recordings each containing two 30-min ECG lead signals (lead A and lead B), sampled at 360Hz. As well only the lead I data were used in the proposed method. In agreement with the AAMI recommended practice, the four recordings with paced beats were removed from the analysis. Five records randomly selected were used to verify the real time application. The remaining recordings were divided into two datasets, with small part of which were used as the training set of the fine-tuning process.
The MIT-BIT Long-term Database is also used in this study for training and verification, which contains 7 long-term ECG recordings (14 to 22 hours each), with manually reviewed beat annotations and sampled at 128Hz. Similarly, the 7 recordings were divided into two datasets, with part used as the fine-tuning training set. A description of the labelled datasets are illustrated in Table 1 .

\subsection{Data Preprocessing}
Similar to the routine of electrocardiography classification task, the workflow con- sists of the stages of prepossessing, processing and the classifying. The prepossessing stage related technologies are not the focus of this study, so the classical methods for prepossessing were adapted, and just a brief introduction of the details would be mentioned.
In the preprocessing stage, filtering algorithms were adapted to remove the arte- fact signals from the ECG signal. The signals include baseline wander, power line interference, and high-frequency noise. For the unlabelled database of ambulatory ECG and the MITBIH LT database, the Lead I data were extracted and a resample from 128Hz to 360Hz procedure was adopted for data consistency.
Before the segmentation procedure from the long-time monitoring ECG signals- Heartbeat Detection: For the heartbeat detection, the MIT-BIH database and un- labelled database, the positions of R waves are determined. The provided fiducial points of R wave had been used as the basis of wave segmentation. The details of the implementation of R wave detection would not be described in this study, and a reference for the R wave detection algorithms had been explored in [58].
In the heartbeat segmentation process, the segmentation program of Laguna [59] was adapted, which also had been validated by other related work [2]. The segmentation process was focus on the Lead I of the recordings. After the segmentation for the ambulatory ECG database, three parts of heartbeat samples listed in Table 1 were acquired for the classification task. As for the pretraining, fine-tuning for our proposed task and comparison, we divided all the samples into three groups: the pretraining group as dataset A, the fine-tuning group as dataset B and test group as dataset C (illustrated in Table 2). Samples are chosen randomly from the original AR and LT database, the details of the sample class would be described in the experiment result analysis. As the algorithm of the deep structure training illustrated, both unsupervised learning and supervised learning are involved in the training process. The pre-training mainly used the unlabelled data to train the autoencoder parts, which only need to set the outputs equal to the inputs. Training data adopted in the unsupervised learning step include the whole samples from the ambulatory electrocardiography database and parts of the MIT-BHI database samples. In the supervised learning step, the MIT-BHI database samples with labels were adopted.

\subsection{Manifold Embedding Methods}

\subsection{Autoencoder Based Feature Extraction}


\section{Results and Discussion}




\section{Conclusion and Future Work}


% An example of a floating figure using the graphicx package.
% Note that \label must occur AFTER (or within) \caption.
% For figures, \caption should occur after the \includegraphics.
% Note that IEEEtran v1.7 and later has special internal code that
% is designed to preserve the operation of \label within \caption
% even when the captionsoff option is in effect. However, because
% of issues like this, it may be the safest practice to put all your
% \label just after \caption rather than within \caption{}.
%
% Reminder: the "draftcls" or "draftclsnofoot", not "draft", class
% option should be used if it is desired that the figures are to be
% displayed while in draft mode.
%
%\begin{figure}[!t]
%\centering
%\includegraphics[width=2.5in]{myfigure}
% where an .eps filename suffix will be assumed under latex, 
% and a .pdf suffix will be assumed for pdflatex; or what has been declared
% via \DeclareGraphicsExtensions.
%\caption{Simulation results for the network.}
%\label{fig_sim}
%\end{figure}

% Note that the IEEE typically puts floats only at the top, even when this
% results in a large percentage of a column being occupied by floats.


% An example of a double column floating figure using two subfigures.
% (The subfig.sty package must be loaded for this to work.)
% The subfigure \label commands are set within each subfloat command,
% and the \label for the overall figure must come after \caption.
% \hfil is used as a separator to get equal spacing.
% Watch out that the combined width of all the subfigures on a 
% line do not exceed the text width or a line break will occur.
%
%\begin{figure*}[!t]
%\centering
%\subfloat[Case I]{\includegraphics[width=2.5in]{box}%
%\label{fig_first_case}}
%\hfil
%\subfloat[Case II]{\includegraphics[width=2.5in]{box}%
%\label{fig_second_case}}
%\caption{Simulation results for the network.}
%\label{fig_sim}
%\end{figure*}
%
% Note that often IEEE papers with subfigures do not employ subfigure
% captions (using the optional argument to \subfloat[]), but instead will
% reference/describe all of them (a), (b), etc., within the main caption.
% Be aware that for subfig.sty to generate the (a), (b), etc., subfigure
% labels, the optional argument to \subfloat must be present. If a
% subcaption is not desired, just leave its contents blank,
% e.g., \subfloat[].


% An example of a floating table. Note that, for IEEE style tables, the
% \caption command should come BEFORE the table and, given that table
% captions serve much like titles, are usually capitalized except for words
% such as a, an, and, as, at, but, by, for, in, nor, of, on, or, the, to
% and up, which are usually not capitalized unless they are the first or
% last word of the caption. Table text will default to \footnotesize as
% the IEEE normally uses this smaller font for tables.
% The \label must come after \caption as always.
%
%\begin{table}[!t]
%% increase table row spacing, adjust to taste
%\renewcommand{\arraystretch}{1.3}
% if using array.sty, it might be a good idea to tweak the value of
% \extrarowheight as needed to properly center the text within the cells
%\caption{An Example of a Table}
%\label{table_example}
%\centering
%% Some packages, such as MDW tools, offer better commands for making tables
%% than the plain LaTeX2e tabular which is used here.
%\begin{tabular}{|c||c|}
%\hline
%One & Two\\
%\hline
%Three & Four\\
%\hline
%\end{tabular}
%\end{table}


% Note that the IEEE does not put floats in the very first column
% - or typically anywhere on the first page for that matter. Also,
% in-text middle ("here") positioning is typically not used, but it
% is allowed and encouraged for Computer Society conferences (but
% not Computer Society journals). Most IEEE journals/conferences use
% top floats exclusively. 
% Note that, LaTeX2e, unlike IEEE journals/conferences, places
% footnotes above bottom floats. This can be corrected via the
% \fnbelowfloat command of the stfloats package.





% if have a single appendix:
%\appendix[Proof of the Zonklar Equations]
% or
%\appendix  % for no appendix heading
% do not use \section anymore after \appendix, only \section*
% is possibly needed

% use appendices with more than one appendix
% then use \section to start each appendix
% you must declare a \section before using any
% \subsection or using \label (\appendices by itself
% starts a section numbered zero.)
%


\appendices
\section{Proof of the First Zonklar Equation}
\lipsum[1-3]
% you can choose not to have a title for an appendix
% if you want by leaving the argument blank
\begin{algorithm}
 \caption{Principal Component Analysis}
 \label{alg1}
 \begin{algorithmic}
 \REQUIRE Data Set $ X = [x_1, x_2, \ldots, x_n]^T$
 \ENSURE $y = x^n$
 \STATE $y \leftarrow 1$
  \end{algorithmic}
 \end{algorithm}

% use section* for acknowledgment
\section*{Acknowledgment}
This work was supported by the National Natural Science Foundation of China, Grant No. 71531004.



% Can use something like this to put references on a page
% by themselves when using endfloat and the captionsoff option.
\ifCLASSOPTIONcaptionsoff
  \newpage
\fi



% trigger a \newpage just before the given reference
% number - used to balance the columns on the last page
% adjust value as needed - may need to be readjusted if
% the document is modified later
%\IEEEtriggeratref{8}
% The "triggered" command can be changed if desired:
%\IEEEtriggercmd{\enlargethispage{-5in}}

% references section

% can use a bibliography generated by BibTeX as a .bbl file
% BibTeX documentation can be easily obtained at:
% http://mirror.ctan.org/biblio/bibtex/contrib/doc/
% The IEEEtran BibTeX style support page is at:
% http://www.michaelshell.org/tex/ieeetran/bibtex/
\bibliographystyle{IEEEtran}
% argument is your BibTeX string definitions and bibliography database(s)
\bibliography{ecgManifold}

% <OR> manually copy in the resultant .bbl file
% set second argument of \begin to the number of references
% (used to reserve space for the reference number labels box)
%\begin{thebibliography}{1}

%\bibitem{IEEEhowto:kopka}
%H.~Kopka and P.~W. Daly, \emph{A Guide to \LaTeX}, 3rd~ed.\hskip 1em plus
 % 0.5em minus 0.4em\relax Harlow, England: Addison-Wesley, 1999.

%\end{thebibliography}

% biography section
% 
% If you have an EPS/PDF photo (graphicx package needed) extra braces are
% needed around the contents of the optional argument to biography to prevent
% the LaTeX parser from getting confused when it sees the complicated
% \includegraphics command within an optional argument. (You could create
% your own custom macro containing the \includegraphics command to make things
% simpler here.)
%\begin{IEEEbiography}[{\includegraphics[width=1in,height=1.25in,clip,keepaspectratio]{mshell}}]{Michael Shell}
% or if you just want to reserve a space for a photo:

\begin{IEEEbiography}[{\includegraphics[width=1in,height=1.25in,clip,keepaspectratio]{YY.pdf}}]{Yan Yan}
received the B.Eng. degree and the MSc in Instrument Engineering at the Harbin Institute of Technology in 2010 and 2012 respectively. 
He worked as research assistance at the Shenzhen Institutes of Advanced Technology, Chinese Academy of Sciences from 2012 to 2014. 
He is working on his PhD in Computer Science started from 2014.
His research interests are digital signal processing, machine learning, and control system.
\end{IEEEbiography}

\begin{IEEEbiography}[{\includegraphics[width=1in,height=1.25in,clip,keepaspectratio]{WL.pdf}}]{Lei Wang}
received the B.Eng. degree in information and control engineering, and the Ph.D. degree in biomedical engineering from Xi’an Jiaotong University, Xi’an, China, in 1995 and 2000, respectively. He was with the University of Glasgow, Glasgow, U.K., and Imperial College London, London, U.K., from 2000 to 2008. He is currently a full professor with the Shenzhen Institutes of Advanced Technology, Chinese Academy of Sciences, Shenzhen, China. He has published over 200 scientific papers, authored four book chapters, and holds 60 patents. His current research interests include body sensor network, digital signal processing, and biomedical engineering
\end{IEEEbiography}

% if you will not have a photo at all:
%\begin{IEEEbiographynophoto}{John Doe}
%%Biography text here.
%\end{IEEEbiographynophoto}

% insert where needed to balance the two columns on the last page with
% biographies
%\newpage

%\begin{IEEEbiographynophoto}{Jane Doe}
%Biography text here.
%\end{IEEEbiographynophoto}

% You can push biographies down or up by placing
% a \vfill before or after them. The appropriate
% use of \vfill depends on what kind of text is
% on the last page and whether or not the columns
% are being equalized.

%\vfill

% Can be used to pull up biographies so that the bottom of the last one
% is flush with the other column.
%\enlargethispage{-5in}



% that's all folks
\end{document}


